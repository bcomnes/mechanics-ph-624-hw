\documentclass{jhwhw}
\usepackage{color}
%\usepackage{amsmath}
\usepackage{mathtools}
\usepackage{graphicx}
\usepackage{hyperref}
\usepackage{braket}
\usepackage{cancel}
%\everymath{\displaystyle}
%\usepackage{bm}
%\usepackage{setspace}
%\usepackage{verbatim}
%\usepackage[lmargin=2.5cm, rmargin=2.5cm,tmargin=3cm,bmargin=2.5cm]{geometry}
%
% Hyperlink styling
\hypersetup{
colorlinks=true,
linkcolor=blue,
urlcolor=blue
}
%
% This allows you to draw a box around things.
% Wrap whateve you want a box around with:
%\Aboxed{Stuff you want boxed}
%\def\@Aboxed#1&#2\ENDDNE{%
%  \settowidth\@tempdima{$\displaystyle#1{}$}%
%  \addtolength\@tempdima{\fboxsep}%
%  \addtolength\@tempdima{\fboxrule}%
%  \global\@tempdima=\@tempdima
%  \kern\@tempdima
%  &
%  \kern-\@tempdima
%  \fcolorbox{red}{yellow}{$\displaystyle #1#2$}
%}
%
% Part of AMS Math
% Lets you create a log-like trace function
\DeclareMathOperator{\Tr}{Tr}
%
% Author and title
\author{Bret Comnes}
\title{QM619 Homework 2}

\begin{document}
\problem{Commutators in the Heisenberg Picture}
Let $\hat{t}$ be the coordinate operator for a free particle in one dimension in the \underline{Heisenberg Picture}.  Evaluate the following commutator:

\begin{equation}
	\left[\hat{x}(t),\hat{x}(0) \right]
\end{equation}

\solution
In the Heisenberg picture, the wave functions are stationary in time, and the operators evolve with time (whereas its the other way around in the S-Picture).  Using the Heisenberg Equation of Motion,

\begin{align}
	\label{eq:heom}
	i \hbar \frac{d \hat{A}_H}{dt} &= [\hat{A}_H, \hat{H}_H] + i \hbar \left( \frac{
	\partial \hat{A}}{\partial t} \right)_H
\end{align}
with the Hamiltonian for the free particle
\begin{align}
	\hat{H} = \frac{\hat{p}^2}{2m}
\end{align}
we can find $\hat{x(t)}$ and then demonstrate the desired commutator relationship.

\begin{align}
	i \hbar \frac{d \hat{x}}{dt} &= [\hat{x}, \hat{H}_H] + i \hbar \left( \frac{
	\partial \hat{x}}{\partial t} \right)_H \\
	i \hbar \frac{d \hat{x}}{dt} &= [\hat{x}, \frac{\hat{p}^2}{2m}] + i \hbar \left( \frac{
	\partial \hat{x}}{\partial t} \right)_H \\
	\frac{d \hat{x}}{dt} &= \frac{1}{i \hbar 2 m }[\hat{x}, \hat{p}^2] + \left( \frac{
	\partial \hat{x}}{\partial t} \right)_H \\
	\frac{d \hat{x}}{dt} &= \frac{1}{i \hbar 2 m }\left([\hat{x}, \hat{p}]\hat{p} + \hat{p}[\hat{x}, \hat{p}] \right) + \left( \frac{
	\partial \hat{x}}{\partial \hat{t}} \right)_H \\
	\frac{d \hat{x}}{dt} &= \frac{1}{i \hbar 2 m }\left(2 i \hbar \hat{p} \right) + \left( \frac{
	\partial \hat{x}}{\partial t} \right)_H \\
	\frac{d \hat{x}}{dt} &= \frac{\hat{p}}{m } + \left( \frac{
	\partial \hat{x}}{\partial t} \right)_H
\end{align}
Since $\left( \frac{\partial \hat{x}}{\partial t} \right)_H$ does not have a direct dependance on time, this term goes to zero, leaving a quick separation of variables to solve for $\hat{x}(t)$.
\begin{align}
	\frac{d \hat{x}}{dt} 
	&= 
	\frac{\hat{p}}{m} \\
	d \hat{x}
	&=
	\frac{p}{m}dt \\
	\int d \hat{x}
	&=
	\int\frac{\hat{p}}{m}dt \\
	\hat{x}(t) + C
	&=
	\frac{\hat{p}}{m}t \\
	\Aboxed{
	\hat{x}(t)
	&=
	\frac{\hat{p}}{m}t + \hat{x}(0) }
\end{align}
We can recognize $C$ as $\hat{x}(0)$, and now we have our time dependent operator, which will allow us to perform the commutation relationship.  
\begin{align}
	[\hat{x}(t),\hat{x}(0)] 
	&=
	\left(\frac{\hat{p}}{m}t + \hat{x}(0)\right)\hat{x}(0)
	-
	\hat{x}(0)\left(\frac{\hat{p}}{m}t + \hat{x}(0)\right)
	\\
	&=
	\frac{\hat{p}}{m}t\hat{x}(0) + \hat{x}^2(0)
	-
	\hat{x}(0)\frac{\hat{p}}{m}t - \hat{x}^2(0)
	\\
	&=
	\frac{\hat{p}}{m}t\hat{x}(0)
	-
	\hat{x}(0)\frac{\hat{p}}{m}t
	\\
	&=
	\frac{t}{m}
	(\hat{p}\hat{x}(0)
	-
	\hat{x}(0)\hat{p})
	\\
	&=
	\frac{t}{m}
	[\hat{p},\hat{x}(0)]
	\\
	\Aboxed{
	[\hat{x}(t),\hat{x}(0)] 
	&=i \hbar\frac{t}{m}
	}
\end{align}

\pagebreak[4]

\problem{Schr\"{o}dinger vs Heisenberg Picture}
Using the \underline{one-dimensional simple harmonic oscillator} as an example, illustrate the differences between the Schr\"{o}dinger (S) and the Heisenberg (H) picture by completing the following:

\underline{\textbf{In the S-Picture:}}

\part
Write down the fundamental equation of motion for the time evolution of a general state vector.
\solution
	We simply pull a Griffiths page one and write down the TDSE:
	\begin{equation}
		\label{eq:tdse}
		i \hbar \frac{\partial}{\partial t}\ket{\psi} = H \ket{\psi}
	\end{equation}
\part
Find an expression for the time evolution operator $\hat{V}(t,t_0)$ and hence determine for the time evolution of a general state vector.
\solution
We start by introducing the time evolution operator such that
\begin{equation}
	\ket{\psi(t)} = \hat{V}(t,t_0)\ket{\psi(t0)}
\end{equation}
which, along with Equation~\eqref{eq:tdse}, we get
\begin{equation}
	i \hbar \frac{\partial}{\partial t} \hat{V}(t,t_0) = H \hat{V}(t,t_0)
\end{equation}
where
\begin{equation}
	\lim_{t \to t_0} \hat{V}(t,t_0) = 1
\end{equation}
At this point we can use separation of variables to solve for an expression for $\hat{V}(t,t_0)$.
\begin{align}
	i \hbar \frac{\partial}{\partial t} \hat{V}(t,t_0) 
	&= 
	H \hat{V}(t,t_0) 
	&&
	\\
	\int i \hbar \frac{\partial(\hat{V}(t,t_0))}{\hat{V}(t,t_0)}
	&= 
	\int H \partial t 
	&&
	\\
	i \hbar \ln (\hat{V}(t,t_0))
	&=
	H(t + c)
	& \text{where }c &= t_0
	\\
	e^{\ln{(\hat{V}(t,t_0))}}
	&=
	e^{H(t - t_0)\frac{1}{i \hbar}}
	&&
	\\
	\Aboxed{
	\hat{V}(t,t_0) = e^{\frac{-i H (t - t_0)}{\hbar}}
	}
	&&
\end{align}
\part
Comment on the time evolution of the dynamic variables $\hat{x}$ and $\hat{p}$.
\solution
In the S-Picture, operators vary with time, where the wave function does not! 

\part
\underline{\textbf{Now In the H-Picture}}

Obtain the equations of motion for the time evolution of the dynamic variables $\hat{x}$ and $\hat{p}$.
\solution
Similar to Problem 1, we can run our operators through the Heisenberg Equation of Motion seen in Equation~\eqref{eq:heom}.  We must note that the Hamiltonian operator is a little different for the SHO compared to the free particle:
\begin{equation}
	\hat{H} = \frac{\hat{p}^2}{2m} + \frac{1}{2} m \omega^2 \hat{x}^2
\end{equation}
Lets find $\hat{x}$ first.  Since there is no explicate time dependance on time for $\hat{x}$, we can argue that $\frac{\partial \hat{x}}{\partial t}$ goes to $0$ right out of the gate.
\begin{align}
	i \hbar \frac{d \hat{x}}{dt} 
	&= 
	[\hat{x}, \hat{H}_H] + i \hbar \cancelto{0}{\left( \frac{
	\partial \hat{x}}{\partial t} \right)_H}
	\\
	\frac{d \hat{x}}{dt} 
	&=
	\frac{1}{i \hbar}
	[\hat{x}, \frac{\hat{p}^2}{2m} + \frac{1}{2} m \omega^2 \hat{x}^2]
	\\
	&=
	\frac{1}{i \hbar} \left(
	\hat{x}\left(\frac{\hat{p}^2}{2m} + \frac{1}{2} m \omega^2 \hat{x}^2 \right) - \left(\frac{\hat{p}^2}{2m} + \frac{1}{2} m \omega^2 \hat{x}^2\right)\hat{x}
	\right)
	\\
	&=
	\frac{1}{i \hbar} \left(
	\frac{\hat{x}\hat{p}^2}{2m} + \cancelto{0}{\frac{1}{2} m \omega^2 \hat{x}^3} - \frac{\hat{p}^2 \hat{x}}{2m} - \cancelto{0}{\frac{1}{2} m \omega^2 \hat{x}^3}
	\right)
	\\
	&=
	\frac{1}{i \hbar 2 m} \left(\hat{x}\hat{p}^2 - \hat{p}^2 \hat{x}\right)
	\\
	&=
	\frac{1}{i \hbar 2 m} (\cancelto{i\hbar}{[\hat{x},\hat{p}]}\hat{p} + \hat{p}\cancelto{i\hbar}{[\hat{x},\hat{p}]})
	& \text{by distributivity :)}
	\\
	\Aboxed{\frac{d \hat{x}}{dt} 
	&=
	\frac{\hat{p}}{m}}
\end{align}

We can do the same thing for $\hat{p}$, making the same argument about $\frac{\partial \hat{p}}{\partial t}$ going to $0$.
\begin{align}
	i \hbar \frac{d \hat{p}}{dt} 
	&= 
	[\hat{p}, \hat{H}_H] + i \hbar \cancelto{0}{\left( \frac{
	\partial \hat{p}}{\partial t} \right)_H}
	\\
	\frac{d \hat{p}}{dt}
	&=
	\frac{1}{i \hbar} [\hat{p}, \frac{\hat{p}^2}{2m} + \frac{1}{2} m \omega^2 \hat{x}^2]
	\\
	&=
	\frac{1}{i \hbar}
	\hat{p} \left(\frac{\hat{p}^2}{2m} + \frac{1}{2} m \omega^2 \hat{x}^2\right)
	- 
	\left(\frac{\hat{p}^2}{2m} + \frac{1}{2} m \omega^2 \hat{x}^2 \right)\hat{p}
	\\
	&=
	\frac{1}{i \hbar} \left(
	 \cancelto{0}{\frac{\hat{p}^3}{2m}}+ \frac{1}{2} m \omega^2 \hat{p} \hat{x}^2
	- 
	\cancelto{0}{\frac{\hat{p}^3}{2m}} - \frac{1}{2} m \omega^2 \hat{x}^2 \hat{p}
	\right)
	\\
	&=
	\frac{m \omega^2}{i \hbar 2} \left(
	\hat{p} \hat{x}^2
	- 
	\hat{x}^2 \hat{p}
	\right)
	\\
	&=
	\frac{m \omega^2}{i \hbar 2} 
	[\hat{p},\hat{x}^2]
	\\
	&=
	\frac{m \omega^2}{i \hbar 2} 
	(\cancelto{-i \hbar}{[\hat{p},\hat{x}]}\hat{x} + \hat{x}\cancelto{-i \hbar}{[\hat{p},\hat{x}]})
	\\
	\frac{d \hat{p}}{dt}
	&=
	- m \omega^2 \hat{x}
	\\
	\Aboxed{\frac{d \hat{p}}{dt}
	&=
	- \beta \hat{x} & \text{where }\beta &= m \omega^2 }
\end{align}

\pagebreak[4]

\part
Solve the above equations in part (d)
\solution
We need to solve the following differential equation:
\begin{align}
    \dot{x} &= \frac{p}{m}
    \\
    \dot{p} &=-\beta x
\end{align}
which happens to be solvable using the general solution to Hook's law:
\begin{equation}
    x(t) = A \sin(\omega t) + B \cos(\omega t)
\end{equation}
where
\begin{equation}
    \label{eq:w}
    \omega^2 = \frac{k}{m} = \frac \beta m
\end{equation}
We can solve for $B$ right off the bat by setting $t = 0$ in $x(t)$:
\begin{align}
    x(0) &= \cancelto{0}{A \sin(0)} + B \cancelto{1}{\cos(0)}
    \\
    x(0) &= B
    \\
    \label{eq:xta}
    x(t) &= A \sin(\omega t) + x_0 \cos(\omega t)
\end{align}
Next we relate the two equations by solving for $\dot{p}$ and integrating:
\begin{align}
    \dot{p} &= -\beta x(t) = -\beta(A \sin(\omega t) + x_0 \cos(\omega t))
    \\
    \int\dot{p} dt &= \int -\beta A \sin(\omega t) dt +  \int x_0 \cos(\omega t))dt
    \\
    \label{eq:pta}
    p &= \frac{\beta A}{\omega} \cos{\omega t} - \frac{\beta x_0}{\omega}\sin{\omega t} + C
\end{align}
Now lets find $A$ by setting $t = 0$ in $p(t)$ and dissolving $C$ into $A$:
\begin{align}
    p(0) &= \frac{\beta A}{\omega} \cancelto{1}{\cos{0}} - \cancelto{0}{\frac{\beta x_0}{\omega}\sin{0}} + C
    \\
    p(0) &= \frac{\beta A}{\omega}
    \\
    \label{eq:a}
    A &= \frac{p_0 \omega}{\beta}
\end{align}
Plugging A~\eqref{eq:a} back into Equation~\eqref{eq:xta} and \eqref{eq:pta}:
\begin{align}
    x(t) &= \frac{p_0 \omega}{\beta} \sin(\omega t) + x_0 \cos(\omega t)
    \\
    p(t) &= p_0 \cos{\omega t} - \frac{\beta x_0}{\omega}\sin{\omega t}
\end{align}
Plugging in Equation~\eqref{eq:w} we get the final solution:
\begin{align}
    \Aboxed{
    x(t) &= \frac{p_0}{\sqrt{\beta m}} \sin(\sqrt{\frac{\beta}{m}} t) + x_0 \cos(\sqrt{\frac{\beta}{m}} t)
    }
    \\
    \Aboxed{
    p(t) &= p_0 \cos{\sqrt{\frac{\beta}{m}} t} - \sqrt{\beta m} x_0 \sin{\sqrt{\frac{\beta}{m}} t}
    }
\end{align}

\part
Comment on both the classical correspondence and the time evolution of the quantum state in this picture.
\solution
The solutions we find here are the same solutions we find in classical mechanics for the SHO!  The wave function does not evolve in time, but rather the operators do.

\pagebreak[4]

\problem{Density Matrix for spin $1/2$ particles}
As shown in the Lecture Notes, the eigenvalue equation for spin-$1/2$ particles can indeed be given by the Pauli matrices as follows:

\begin{align}
	\label{eq:smatrix}
	S\ket{\chi} &= \frac{\hbar}{2}\sigma \ket{\chi}
\end{align}
where
\begin{align}
	\sigma_x &= 
	\begin{pmatrix}
		0	&	1	\\
		1	&	0
	\end{pmatrix}
	&
	\sigma_x &= 
	\begin{pmatrix}
		0	&	-i	\\
		i	&	0
	\end{pmatrix}
	&
	\sigma_x &= 
	\begin{pmatrix}
		1	&	0	\\
		0	&	-1
	\end{pmatrix}
\end{align}

\part
Show that the above spin eigenvectors along each of the x and z directions with eigenvalues $+ \frac{\hbar}{2}$ and $-\frac{\hbar}{2}$ are given by:
\begin{align}
	\ket{\chi_z +} &\equiv 
	\ket{+} = 
	\begin{pmatrix}
		1 \\
		0
	\end{pmatrix}
	&
	\ket{\chi_z -} &\equiv 
	\ket{-} = 
	\begin{pmatrix}
		0 \\
		1
	\end{pmatrix}
	\\
	\ket{\chi_x +} &= 
	\frac{1}{\sqrt{2}}(\ket{+} + \ket{-}) = 
	\frac{1}{\sqrt{2}}
	\begin{pmatrix}
		1 \\
		1
	\end{pmatrix}
	&
	\ket{\chi_x -} &= 
	\frac{1}{\sqrt{2}}(\ket{+} - \ket{-}) = 
	\frac{1}{\sqrt{2}}
	\begin{pmatrix}
		1 \\
		-1
	\end{pmatrix}
	\label{eq:xsmatrix}
\end{align}
\solution
Since we know the eigenvalues already, all we have to do is solve a really basic eigenvalue problem and then we will have shown how to get our spin eigenvectors.  

In the Z direction, the eigenvalue problem can be reduced to the following form:
\begin{equation}
	\begin{pmatrix}
		1	&	0	\\
		0	&	-1
	\end{pmatrix}
	\begin{pmatrix}
		a	\\
		b
	\end{pmatrix}
	=
	\lambda
	\begin{pmatrix}
		a	\\
		b
	\end{pmatrix}
\end{equation}
Solving the characteristic equation
\begin{align}
	\left|\underline{\underline{M}} - \lambda \underline{\underline{I}}\right| &= 0
	\\
	\begin{vmatrix}
		1-\lambda	&	0		\\
		0			&	-1 - \lambda
	\end{vmatrix}
	&=
	0
	\\
	(1 -\lambda)(-1 - \lambda) = -1 - \lambda^2 &= 0
	\\
	\lambda &= \pm 1
\end{align}
The sign on $\lambda$ represents our spin up and spin down state.  

When $\lambda = +1$:
\begin{align}
	&&\begin{pmatrix}
		1	&	0	\\
		0	&	-1
	\end{pmatrix}
	\begin{pmatrix}
		a	\\
		b
	\end{pmatrix}
	&=
	+1
	\begin{pmatrix}
		a	\\
		b
	\end{pmatrix}
	&&
	\\
	 a &= +a &&& -b &= b
\end{align}
These results, under normalization, yield:
\begin{align}
	\begin{pmatrix}
		a	\\
		b
	\end{pmatrix}_+
	&=
	\begin{pmatrix}
		1	\\
		0
	\end{pmatrix}
\end{align}

Similarly when $\lambda = -1$
\begin{align}
	&&\begin{pmatrix}
		1	&	0	\\
		0	&	-1
	\end{pmatrix}
	\begin{pmatrix}
		a	\\
		b
	\end{pmatrix}
	&=
	-1
	\begin{pmatrix}
		a	\\
		b
	\end{pmatrix}
	&&
	\\
	 a &= -a &&& -b &= -b
\end{align}
Again, under normalization, these results yield our eigenvectors:
\begin{align}
	\begin{pmatrix}
		a	\\
		b
	\end{pmatrix}_-
	&=
	\begin{pmatrix}
		0	\\
		1
	\end{pmatrix}
\end{align}
Putting these back into Equation~\eqref{eq:smatrix}, yields the following spin eigenvectors in the $z$ direction.
\begin{align}
	\Aboxed{
	\ket{\chi_z +} &\equiv 
	\ket{+} = 
	\begin{pmatrix}
		1 \\
		0
	\end{pmatrix}}
	&
	\Aboxed{
	\ket{\chi_z -} &\equiv 
	\ket{-} = 
	\begin{pmatrix}
		0 \\
		1
	\end{pmatrix}}
\end{align}

Similarly in the x-direction, we solve our characteristic equation:
\begin{align}
	\begin{pmatrix}
		0	&	1	\\
		1	&	0
	\end{pmatrix}
	\begin{pmatrix}
		a	\\
		b
	\end{pmatrix}
	&=
	\lambda
	\begin{pmatrix}
		a	\\
		b
	\end{pmatrix}
	\\
	\left|\underline{\underline{M}} - \lambda \underline{\underline{I}}\right| &= 0
	\\
	\begin{vmatrix}
		0-\lambda	&	1		\\
		1			&	0 - \lambda
	\end{vmatrix}
	&=
	0
	\\
	(-\lambda)(-\lambda)-1 &= 0
	\\
	\lambda &= \pm 1
\end{align}
We again find our eigenvectors for the given eigenvalues.

For $\lambda = +1$:
 \begin{align}
 	\begin{pmatrix}
 		0	&	1	\\
 		1	&	0
 	\end{pmatrix}
 	\begin{pmatrix}
 		a	\\
 		b
 	\end{pmatrix}
 	&=
 	+1
 	\begin{pmatrix}
 		a	\\
 		b
 	\end{pmatrix}
 	\\
 	 a &= b
 \end{align}
Again, we find the the eigenvector
\begin{align}
 	\begin{pmatrix}
 		a	\\
 		b
 	\end{pmatrix}_+
 	&=
 	\begin{pmatrix}
 		a	\\
 		a
 	\end{pmatrix}
 \end{align}
 however under normalization
 \begin{align}
 	\sqrt{a^2+a^2} &= 1
	\\
	a = \frac{1}{\sqrt{2}}
 \end{align}
 resulting in our desired eigenvector seen in Equation~\eqref{eq:xsmatrix}
 \begin{align}
	 \Aboxed{
	\ket{\chi_x +} = 
	\frac{1}{\sqrt{2}}(\ket{+} + \ket{-}) = 
	\frac{1}{\sqrt{2}}
	\begin{pmatrix}
		1 \\
		1
	\end{pmatrix}}
 \end{align}

Now for $\lambda = -1$:
\begin{align}
	\begin{pmatrix}
		0	&	1	\\
		1	&	0
	\end{pmatrix}
	\begin{pmatrix}
		a	\\
		b
	\end{pmatrix}
	&=
	-1
	\begin{pmatrix}
		a	\\
		b
	\end{pmatrix}
	\\
	\begin{pmatrix}
		b	\\
		a
	\end{pmatrix}
	&=
	\begin{pmatrix}
		-a	\\
		-b
	\end{pmatrix}
	\\
	\begin{pmatrix}
		a	\\
		b
	\end{pmatrix}_-
	&=
	\begin{pmatrix}
		a	\\
		-a
	\end{pmatrix}
\end{align}
Under normalization:
\begin{align}
	\sqrt{(a)^2+(-a)^2} &= 1 \\
	a = \frac{1}{\sqrt{2}}
\end{align}
which brings us to the final eigenvector we seek:
\begin{align}
	\Aboxed{
	\ket{\chi_x -} &= 
	\frac{1}{\sqrt{2}}(\ket{+} - \ket{-}) = 
	\frac{1}{\sqrt{2}}
	\begin{pmatrix}
		1 \\
		-1
	\end{pmatrix}
	}
\end{align}
\pagebreak[4]

\part
Show that for a pure ensemble of a completely polarized beam of $1/2$ particles with spin up along the z-direction, i.e. $\ket{\chi_z +} = \ket{+}$, the density matrix can be expressed as:
\begin{equation}
	\rho_z =
	\begin{pmatrix}
		1	&	0	\\
		0	&	0
	\end{pmatrix}
\end{equation}
\solution
We need to use the definition of the density operator from equation (32.7) in the qm-619 notes,
\begin{equation}
	\label{eq:density}
	\rho \equiv \sum \limits_i f_i \rho^{(i)}
\end{equation}
where
\begin{equation}
	\rho^{(i)} \equiv \ket{\Psi^{(i)}}\bra{\Psi^{(i)}}
\end{equation}
and
\begin{equation}
	f_i \equiv \frac{N_i}{N}
\end{equation}
where $N$ is the total number of identical particles making up the ensemble and $N_i$ is the number of particles in state $i$ within the ensemble.

For an ensemble of $1/2$ particles all with spin up, Equation~\eqref{eq:density} reduces to:
\begin{align}
	\rho = \rho^{(z+)} &= \ket{+}\bra{+}
	\\
	\rho^{(z)}
	&=
	\begin{pmatrix}
		1	\\
		0
	\end{pmatrix}
	\begin{pmatrix}
		1	&	0
	\end{pmatrix}
	=
	\begin{pmatrix}
		1	&	0	\\
		0	&	0
	\end{pmatrix}
	\\
	\Aboxed{	\rho^{(z)}
	&=
	{\begin{pmatrix}
		1	&	0	\\
		0	&	0
	\end{pmatrix}}}
\end{align}

\part
Same problem as in (b), but now for a pure state $\ket{\chi_x +} = \frac{1}{\sqrt{2}}(\ket{+} + \ket{-})$, show that
\begin{equation}
	\rho_x = \frac{1}{2}
	\begin{pmatrix}
		1	&	1	\\
		1	&	1
	\end{pmatrix}
\end{equation}
\solution
This is very similar to the previous part. 
\begin{align}
	\rho = \rho^{(x)} &= \ket{\chi_x+}\bra{\chi_x+}
	\\
	\rho^{(x)} &= \frac{1}{\sqrt{2}}\frac{1}{\sqrt{2}}
	\begin{pmatrix}
		1	\\
		1
	\end{pmatrix}
	\begin{pmatrix}
		1	&	1
	\end{pmatrix}
	\\
	\Aboxed{\rho^{(x)} &= \frac{1}{2}
	{\begin{pmatrix}
		1	&	1	\\
		1	&	1
	\end{pmatrix}}}
\end{align}

\part
Comment on the values for $\Tr(\rho)$ and $\Tr(\rho^2)$ in (b) and (c), respectively.
\solution
\begin{align}
\Aboxed
{\Tr(\rho_{(z)})	&= 
	\Tr{{\begin{pmatrix}
		1	&	0	\\
		0	&	0
	\end{pmatrix}}} = 1+0= 1}
	\\
	\Aboxed{\Tr(\rho^2_{(z)})	&= 
	\Tr{
	{\begin{pmatrix}
		1	&	0	\\
		0	&	0
	\end{pmatrix}}}^2 
	= 
	\Tr{{\begin{pmatrix}
		1	&	0	\\
		0	&	0
	\end{pmatrix}}} = 1+0 = 1}
	\\
	\Aboxed{\Tr(\rho_{(x)})	&= 
		\Tr\frac{1}{2}{{\begin{pmatrix}
			1	&	1	\\
			1	&	1
		\end{pmatrix}}} = \frac{1}{2} + \frac{1}{2} = 1}
	\\
	\Aboxed{\Tr(\rho^2_{(x)})	&= 
	\Tr \left(
	\left(\frac{1}{2}{{\begin{pmatrix}
		1	&	1	\\
		1	&	1
	\end{pmatrix}}}\right)^2\right) 
	=
	\Tr 
	{{\begin{pmatrix}
		\frac{1}{2}	&	\frac{1}{2}	\\
		\frac{1}{2}	&	\frac{1}{2}
	\end{pmatrix}}}
	\cdot
	{{\begin{pmatrix}
		\frac{1}{2}	&	\frac{1}{2}	\\
		\frac{1}{2}	&	\frac{1}{2}
	\end{pmatrix}}}
	=
	\Tr{{\begin{pmatrix}
		\frac{1}{2}	&	\frac{1}{2}	\\
		\frac{1}{2}	&	\frac{1}{2}
	\end{pmatrix}}}
	=
	\frac{1}{2}+\frac{1}{2}
	= 1}
\end{align}

They are all 1!

\part
For a completely unpolarized beam of spin $1/2$ particles, show that
\begin{align}
	\rho_x &=
	\frac{1}{2}
	\begin{pmatrix}
		1	&	0	\\
		0	&	1
	\end{pmatrix}
	&
	\rho_x &=
	\frac{1}{4}
	\left[
	\begin{pmatrix}
		1	&	1	\\
		1	&	1
	\end{pmatrix}
	+
	\begin{pmatrix}
		1	&	-1	\\
		-1	&	1
	\end{pmatrix}
	\right]
	&
	\rho_z &= \rho_x
\end{align}
\solution
For an unpolarized beam, we have an equal chance of finding particles in the spin up or spin down state, so $\frac{N_i}{N} = \frac{1}{2}$ for $i = \pm$.  Looking at $\rho^{(z\pm)}$ first:
\begin{align}
	\rho = \frac{1}{2}\rho^{(z+)} + \frac{1}{2}\rho^{(z-)} &= \frac{1}{2}(( \ket{+}\bra{+}) + (\ket{-}\bra{-} ))
	\\
	&=
	\frac{1}{2} \left( 
	\begin{pmatrix}
		1	\\
		0
	\end{pmatrix}
	\begin{pmatrix}
		1	&	0
	\end{pmatrix}
	+
	\begin{pmatrix}
		0	\\
		1
	\end{pmatrix}
	\begin{pmatrix}
		0	&	1
	\end{pmatrix}
	\right)
	=
	\frac{1}{2}\left(
	\begin{pmatrix}
		1	&	0	\\
		0	&	0
	\end{pmatrix}
	+
	\begin{pmatrix}
		0	&	0	\\
		0	&	1
	\end{pmatrix}
	\right)
	\\
	\Aboxed{	\rho^{(z\pm)}
	&=
	\frac{1}{2}
	{\begin{pmatrix}
		1	&	0	\\
		0	&	1
	\end{pmatrix}}}
	\label{eq:pzi}
\end{align}
Similarly looking $\rho^{x\pm}$:
\begin{align}
	\rho = \frac{1}{2}\rho^{(x+)} + \frac{1}{2}\rho^{(x-)} &= \frac{1}{2}(( \ket{\chi_x+}\bra{\chi_x+}) + (\ket{\chi_x-}\bra{\chi_x-} )) 
	\\
	&=
	\frac{1}{2}
	\frac{1}{\sqrt{2}}
	\frac{1}{\sqrt{2}}
	\left(
	{\begin{pmatrix}
		1	\\
		1
	\end{pmatrix}}
	{\begin{pmatrix}
		1	&	1
	\end{pmatrix}}
	+
	{\begin{pmatrix}
		1	\\
		-1
	\end{pmatrix}}
	{\begin{pmatrix}
		1	&	-1
	\end{pmatrix}}
	\right)
	\\
	\Aboxed{
	\rho^{x\pm}
	&=
	\frac{1}{4}
	\left[
	{\begin{pmatrix}
		1	&	1	\\
		1	&	1
	\end{pmatrix}}
	+
	{\begin{pmatrix}
		1	&	-1	\\
		-1	&	1
	\end{pmatrix}}
	\right]}
	\label{eq:pxi}
\end{align}
Which allows us to conclude
\begin{equation}
	\boxed{\rho^{z\pm} = \rho^{x\pm}}
\end{equation}

\part
Show also that in the case of (e): the quantum ensemble average of each of the spin components of the beam is zero, i.e.: 
\begin{equation}
	\braket{\bar{S_x}}=\braket{\bar{S_y}}=\braket{\bar{S_z}}=0
\end{equation}
\solution
Calculating the expectation value of $S$ is simple.  All we need to perform is $\braket{\chi|\hat{S}|\chi}$.
\begin{align}
	\braket{\chi|\hat{S_z}|\chi} &= \braket{+|\hat{S}|+} + \braket{-|\hat{S}|-}
\end{align}
For example in the $z$ direction
\begin{align}
	\braket{\bar{S}_z} &= \sum \limits_i f_i \Tr (\hat{S}_z \rho^{(i)})
	\\
	&=
	\frac{\hbar}{2}\left(
	f_+ \Tr{\left( 
	{\begin{pmatrix}
		1	&	0	\\
		0	&	-1
	\end{pmatrix}}
	{\begin{pmatrix}
		1	\\
		0
	\end{pmatrix}}
	{\begin{pmatrix}
		1	&	0
	\end{pmatrix}}
	\right)}
	+
	f_- \Tr{\left( 
	{\begin{pmatrix}
		1	&	0	\\
		0	&	-1
	\end{pmatrix}}
	{\begin{pmatrix}
		0	\\
		1
	\end{pmatrix}}
	{\begin{pmatrix}
		0	&	1
	\end{pmatrix}}
	\right)}
	\right)
	\\
	&=
	\frac{\hbar}{2}\left(
	f_+\Tr{\begin{pmatrix}
		1	&	0	\\
		0	&	0
	\end{pmatrix}}
	+
	f_-\Tr{\begin{pmatrix}
		0	&	0	\\
		0	&	-1
	\end{pmatrix}}
	\right)
	\\
	&=
	\frac{\hbar}{2}\left(
	f_+
	-
	f_-
	\right)
	\\
	\Aboxed{\braket{\bar{S}_z} &= 0}
\end{align}
since the beam is unpolarized so $f_- = f_+$.  

We could continue this way, but finding $\braket{\bar{S}_x}$ and $\braket{\bar{S}_y}$ is a bit more involved.  Instead lets notice that the density matrices $\rho^{x\pm}$ and $\rho^{z\pm}$ both equal $\frac{1}{2}\mathbf{I}$, and when running that through the double average of $S$, due to the known eigenvalues $\pm\frac{\hbar}{2}$, that summing this will always result in 0.  We already demonstrated this in Equation~\eqref{eq:pzi} and \eqref{eq:pzi} for $\braket{\bar{S}_x}$ and $\braket{\bar{S}_z}$.  This leaves us to confirm $\rho^{(y\pm)}= \frac{1}{2}\mathbf{I}$.  Looking up the spin-$1/2$ eigenstates in the y-direction\cite{mc}:
\begin{align}
    \ket{\chi_y +} &=
    \frac{1}{\sqrt{2}}
	\begin{pmatrix}
		1 \\
		+i
	\end{pmatrix}
    &
    \ket{\chi_y -} &=
    \frac{1}{\sqrt{2}}
	\begin{pmatrix}
		1 \\
		-i
	\end{pmatrix}
\end{align}
we can quickly make said confirmation assuming:
\begin{align}
	\rho^{(y\pm)} 
	= 
	\frac{1}{2}\rho^{(y+)} + \frac{1}{2}\rho^{(y-)} &= \frac{1}{2}(( \ket{\chi_y+}\bra{\chi_y+}) + (\ket{\chi_y-}\bra{\chi_y-} )) 
	\\
	&=
	\frac{1}{2}
	\frac{1}{\sqrt{2}}
	\frac{1}{\sqrt{2}}
	\left(
	{\begin{pmatrix}
		1	\\
		i
	\end{pmatrix}}
	{\begin{pmatrix}
		1	&	-i
	\end{pmatrix}}
	+
	{\begin{pmatrix}
		1	\\
		-i
	\end{pmatrix}}
	{\begin{pmatrix}
		1	&	i
	\end{pmatrix}}
	\right)
	\\
	\rho^{y\pm}
	&=
	\frac{1}{4}
	\left[
	{\begin{pmatrix}
		1	&	-i	\\
		i	&	1
	\end{pmatrix}}
	+
	{\begin{pmatrix}
		1	&	i	\\
		-i	&	1
	\end{pmatrix}}
	\right]
	\\
	\Aboxed{\rho^{y\pm}
	&=
	\frac{1}{2}
	{\begin{pmatrix}
		1	&	0	\\
		0	&	1
	\end{pmatrix}}
	\label{eq:pyi}}
\end{align}
So thats it! $\rho^{(y)}=\frac{1}{2}\mathbf{I}$, and thus $\braket{\bar{S}_y} = 0$, but we can go ahead and do it anyway because I already typed it in before correcting the above statement.  

\begin{align}
	\braket{\bar{S}_y} &= \sum \limits_i f_i \Tr (\hat{S}_y \rho^{(i)})
	\\
	&=
	\frac{\hbar}{2}\left(
	f_+ \Tr{\left( 
	{\begin{pmatrix}
		0	&	-i	\\
		i	&	0
	\end{pmatrix}}
	\frac{1}{\sqrt{2}}\frac{1}{\sqrt{2}}
	{\begin{pmatrix}
		1	\\
		i
	\end{pmatrix}}
	{\begin{pmatrix}
		1	&	-i
	\end{pmatrix}}
	\right)}
	+
	f_- \Tr{\left( 
	{\begin{pmatrix}
	    0	&	-i	\\
	    i	&	0
	\end{pmatrix}}
	\frac{1}{\sqrt{2}}\frac{1}{\sqrt{2}}
	{\begin{pmatrix}
		1	\\
		-i
	\end{pmatrix}}
	{\begin{pmatrix}
		1	&		i
	\end{pmatrix}}
	\right)}
	\right)
	\\
	&=
	\frac{\hbar}{2}\frac{1}{2}\left(
	f_+\Tr{\left( \begin{pmatrix}
		0	&	-i	\\
		i	&	0
	\end{pmatrix}
	{\begin{pmatrix}
	   1    &   -i   \\
	   i    &   1
	\end{pmatrix}}
	\right)}
	+
	f_-\Tr{\left(\begin{pmatrix}
		0	&	-i	\\
		i	&	0
	\end{pmatrix}
	{\begin{pmatrix}
	   1    &   i   \\
	   -i    &   1
	\end{pmatrix}}
	\right)}
	\right)
	\\
	&=
	\frac{\hbar}{4}\left(
	f_+
	\cancelto{0}
	{\Tr{\left( \begin{pmatrix}
		1	&	-i	\\
		i	&	1
	\end{pmatrix}
	\right)}
	+
	f_-
	\Tr{\left(\begin{pmatrix}
		-1	&	-i	\\
		i	&	-1
	\end{pmatrix}
	\right)}}
	\right)
	\\
	\Aboxed{\braket{\bar{S}_y} &= 0}
\end{align}
So $\braket{\bar{S}_y}$ goes to 0 this way too.  And we can also argue that once we know z and x we don't care about y since its going to be the same or something.

\part
For a beam of spin $1/2$ particles in a mixed state with $60\%$ ``spin up'' along the z axis (i.e. in the $\ket{\chi_z +}$ state) and $40\%$ in the $\ket{\chi_z -}$ state, find the density matrix and show that $\Tr(\rho^2)<1$ in this case.
\solution
Using Equation~\eqref{eq:density}
\begin{align}
    \rho &= (0.6) \ket{\chi_z +}\bra{\chi_z +} + (0.4)\ket{\chi_z -}\bra{\chi_z -}
    \\
    &=
    \left(
    (0.6) 
	\begin{pmatrix}
		1	\\
		0
	\end{pmatrix}
	\begin{pmatrix}
		1	&	0
	\end{pmatrix}
	+
	(0.4)
	\begin{pmatrix}
		0	\\
		1
	\end{pmatrix}
	\begin{pmatrix}
		0	&	1
	\end{pmatrix}
	\right)
	\\
    &=
    \left(
    (0.6) 
	\begin{pmatrix}
		1   &   0	\\
		0   &   0
	\end{pmatrix}
	+
	(0.4)
	\begin{pmatrix}
		0	&   0   \\
		0   &   1
	\end{pmatrix}
	\right)
	\\
	\rho &=
	\begin{pmatrix}
	   0.6  &   0   \\
	   0    &   0.4
	\end{pmatrix}
	\\
	\rho^2 &=
	\begin{pmatrix}
	   0.6  &   0   \\
	   0    &   0.4
	\end{pmatrix}
	\cdot
	\begin{pmatrix}
	   0.6  &   0   \\
	   0    &   0.4
	\end{pmatrix}
	=
	\begin{pmatrix}
	   0.36 &   0   \\
	   0    &   0.16
	\end{pmatrix}
	\\
	\Aboxed{\Tr(\rho^2) &= 0.52}
\end{align}

\pagebreak[4]

\problem{Berry Phase}
For the example discussed in class with a spin $1/2$ particle in a magnetic field $(\vec{B})$ of constant magnitude precessing slowly about the z-axis:

\part
Show that the Berry phase associated with the spin state aligning \underline{opposite} to $\vec{B}$ with the eigenstate given in eq. (34.21) is by:
\begin{equation}
	\gamma_-(T) = \frac{1}{2}\Omega-2\pi
\end{equation}
\solution
We will be following along with the procedure done in the notes on page [H-16], except calculating $\gamma_-(T)$ instead.
Looking at Equation (34.21) from the notes:
\begin{align}
    \gamma_-
    &=
    \begin{pmatrix}
        \sin{\frac{\alpha}{2}}  \\
        -\cos{\frac{\alpha}{2}}e^{i{\omega t}}
    \end{pmatrix}
\end{align}
where $\omega t = \psi$.  Now we find $\vec{A}_+$:
\begin{align}
    \label{eq:am}
    \vec{A}_+ &= i \braket{\psi_-|\vec{\nabla}_r \psi_-}
\end{align}
In spherical coordinates:
\begin{align}
    \vec{\nabla}_r &=
    \hat{e}_r\frac{\partial}{\partial r}
    +
    \hat{e}_\alpha\frac{1}{r}\frac{\partial}{\partial \alpha}
    +
    \hat{e}_\psi\frac{1}{r \sin{\alpha}}\frac{\partial}{\partial \psi}
\end{align}
so that Equation~\eqref{eq:am} turns into:
\begin{align}
     \vec{A}_+
     &=
     i
     \begin{pmatrix}
        \sin{\frac{\alpha}{2}}  &   -\cos{\frac{\alpha}{2}}e^{-i \psi}
     \end{pmatrix}
     \frac{1}{r}
     \begin{pmatrix}
        \frac{1}{2}\cos{\frac{\alpha}{2}} \\
        \frac{1}{2}e^{i \psi}\sin{\frac{\alpha}{2}}
     \end{pmatrix}
     \hat{e}_\alpha
     +
     \frac{i}{r \sin{\alpha}}
     \begin{pmatrix}
        \sin{\frac{\alpha}{2}}  &   -\cos{\frac{\alpha}{2}}e^{-\psi}
     \end{pmatrix}
     \begin{pmatrix}
        0   \\
        -i e^{i \psi} \cos{\frac{\alpha}{2}}
     \end{pmatrix}
     \hat{e}_{\psi}
     \\
     &=
     \left(
     \sin{\frac{\alpha}{2}}\frac{1}{r}\frac{1}{2}\cos{\frac{\alpha}{2}}-\cos{\frac{\alpha}{2}}\cancelto{1}e^{-i \psi}\frac{1}{r}\frac{1}{2}\cancelto{1}e^{i \psi}\sin{\frac{\alpha}{2}}
     \right) \hat{e}_\alpha
     +
     \frac{i}{\sin{\alpha}}
     \left(
     \cos{\frac{\alpha}{2}}\cancelto{1}e^{-i\psi}i \cancelto{1}e^{i \psi}\cos{\frac{\alpha}{2}}
     \right)\hat{e}_{\psi}
     \\
     &=
     \cancelto{0}{\left(
     \sin{\frac{\alpha}{2}}\frac{1}{r}\frac{1}{2}\cos{\frac{\alpha}{2}}-\cos{\frac{\alpha}{2}}\frac{1}{r}\frac{1}{2}\sin{\frac{\alpha}{2}}
      \right)} \hat{e}_\alpha
      +
      \frac{i}{\sin{\alpha}}(i\cos^2{\frac{\alpha}{2}})\hat{e}_{\psi}
      \\
      \vec{A}_+
      &=
       \frac{-\cos^2{\frac{\alpha}{2}}}{\sin{\alpha}}\hat{e}_{\psi}
\end{align}
Remembering that
\begin{equation}
    \sin{\alpha} = 2 \cos{\frac{\alpha}{2}}\sin{\frac{\alpha}{2}}
\end{equation}
so that we finally get
\begin{align}
    \Aboxed{
    \vec{A}_+
      &=
       \frac{-\cos{\frac{\alpha}{2}}}{2 r \sin{\frac{\alpha}{2}}}\hat{e}_{\psi}
       }
\end{align}
At this point we are ready to find $\gamma_-(T)$.  Following Equation~[34.11]
\begin{align}
    \gamma_-(T)
    &=
    \oint \vec{A}_-\cdot d \vec{R} = \oint \vec{A}_-\cdot d \vec{B}
    \\
    &=
    -\frac{1}{2B_0} 
    \frac
    {\cos{\frac{\alpha}{2}}}
    {\sin{\frac{\alpha}{2}}} 
    \int_{0}^{2\pi}
    B_{0} \sin{\alpha} d \psi
    \\
    &=
    -\frac{1}{2} 
    \frac
    {\cos{\frac{\alpha}{2}}}
    {\sin{\frac{\alpha}{2}}} 
    \sin{\alpha} 2 \pi
    \\
    &=
    -
    \frac
    {\cos{\frac{\alpha}{2}}}
    {\sin{\frac{\alpha}{2}}} 
    2\cos{\frac{\alpha}{2}}\sin{\frac{\alpha}{2}} (2 \pi)
    \\
    &=
    -
    {\cos^2{\frac{\alpha}{2}}}(2 \pi)
    \\
    &=
    -
    {\frac{1}{2}-\frac{1}{2}\cos{\alpha}}(2 \pi)
    \\
    \Aboxed{
    \gamma_-(T)
    &=
    -({1+\cos{\alpha}})\pi
    }
\end{align}
To get this to its final form, lets solve for $\gamma_-(T)$ in terms of $\Omega$
\begin{align}
    \Omega &= (1 - \cos{\alpha}) 2 \pi
    \\
    \Omega / 2 - 2 \pi &= (1 - \cos{\alpha}) \pi - 2\pi
    \\
    \Omega / 2 - 2 \pi &= -({1+\cos{\alpha}})\pi
    \\
    \Aboxed{
    \gamma_-(T)
    &=
    \Omega / 2 - 2 \pi
    }
\end{align}
\pagebreak[4]


\part
Now consider the more general situation when $\vec{B}$ does not precess with a constant azimuthal angle.  Let us generalize the ``motion of the tip'' of $\vec{B}$ to ``sweep'' out an arbitrary closed loop on the surface of a sphere as shown.  The eigenstate $\psi_+$ is given by:
\begin{equation}
	\psi_+ = 
	\begin{pmatrix}
		\cos{\frac{\alpha}{2}} \\
		\sin{\frac{\alpha}{2}}e^{i \varphi}
	\end{pmatrix}
\end{equation} 
with $\alpha(t)$ and $\varphi(t)$ assuming instantaneous values.  Using eq. (34.13) from the lecture notes, show that the Berry phase calculated for this case over one complete cycle is still given by (34.23), i.e.
\begin{equation}
	\gamma_+(T) = -\frac{1}{2}\Omega
\end{equation}
Where $\Omega$ is the solid angle subtended by the surface on the sphere swept out by $\vec{B}$.
\solution
Lets do this!
\begin{align}
    \psi_+ &=
    \begin{pmatrix}
        \cos{\frac{\alpha}{2}} \\
        \sin{\frac{\alpha}{2}}e^{i \psi}
    \end{pmatrix}
    \\
    \vec{A}_+
    &=
    -\frac{1}{2 B_0} \tan{\frac{\alpha}{2}} \hat{e}_\psi
\end{align}
Equation~[34.13] states:
\begin{align}
    \gamma_n
    &=
    \int_s
    (
    \vec{\nabla}_r \times \vec{A}_n
    )
    \cdot
    d \vec{S}
    \label{eq:gamn}
\end{align}
Looking up $\vec{\nabla}_r \times \vec{A}_n$ in Griffiths, dropping terms that do not have $\hat{e}_\psi$, and swapping $r$ for $B_0$  we get:
\begin{align}
    \vec{\nabla}_r \times \vec{A}_n
    &=
    \frac{1}{r \sin{\alpha}}
    \left[
    \frac{\partial}{\partial \alpha} (\sin{(\alpha)}A_\psi)
    \right]\hat{e}_r
    +
    \frac{1}{r}
    \left[
    -\frac{\partial}{\partial r}(r A_\psi)
    \right]\hat{e}_\alpha
    \\
    &=
    \frac{1}{B_0 \sin{\alpha}}
    \left[
    \frac{\partial}{\partial \alpha} (\sin{(\alpha)}\left(\frac{-1}{2 B_0} \tan{\frac{\alpha}{2}}\right))
    \right]\hat{e}_{B_0}
    +
    \frac{1}{B_0}
    \left[
    -\frac{\partial}{\partial B_0}(B_0 \left(\frac{-1}{2 B_0} \tan{\frac{\alpha}{2}}\right))
    \right]\hat{e}_\alpha
    \\
    &=
    \frac{1}{B_0 \sin{\alpha}}
    \frac{-1}{2 B_0}
    \frac{\partial}{\partial\alpha}
    (
    \sin{\alpha}\tan{\frac{\alpha}{2}}
    )\hat{e}_{B_0}
    +
    \cancelto{0}
    {
    \frac{1}{B_0}
    \tan{\frac{\alpha}{2}}
    \frac{\partial}{\partial B_0}\frac{1}{2}\hat{e}_\alpha}
    \\
    &=
    \frac{-1}{2 B_0 \sin{\alpha}} \frac{\partial}{\partial \alpha}
    (
    \sin{\alpha}\tan{\frac{\alpha}{2}}
    )
    \hat{e}_{B_0}
    \\
    &=
    \frac{-1}{2 B_0 \sin{\alpha}}
    \left(
    \frac{1}{2}\sec^2{(\frac{\alpha}{2})}\sin{\alpha}+\cos{\alpha}\tan{\frac{\alpha}{2}}
    \right)
    \hat{e}_{B_0}
    \\
    \label{eq:res}
    &=
    \frac{-1}{2 B_0}
    \left(
    \frac{1}{2}\sec^2{(\frac{\alpha}{2})}+\cot{\alpha}\tan{\frac{\alpha}{2}}
    \right)
    \hat{e}_{B_0}
\end{align}
Remembering our half angle theorems
\begin{align}
    \frac{1}{\cos^2{\frac{a}{2}}} &= \frac{2}{1 + \cos{\alpha}}
    \\
    \tan{a}{2}&= \frac{\sin{a}}{1+\cos{a}}
\end{align}
we can continue reducing Equation~\eqref{eq:res}
\begin{align}
    \vec{\nabla}_r \times \vec{A}_n
    &=
    \frac{-1}{2 B_0}
    \left(
    \frac{1}{2}
    \frac{1}{\cos^2{\frac{\alpha}{2}}}
    +
    \frac{\cos{\alpha}}{\sin{\alpha}}
    \tan{\frac{\alpha}{2}}
    \right)\hat{e}_{B_0}
    \\
    &=
    \frac{-1}{2 B_0}
    \left(
    \frac{1}{(1 + \cos{\alpha})}
    +
    \cos{\alpha}
    \frac{1}{1 + \cos{\alpha}}
    \right)\hat{e}_{B_0}
    \\
    \Aboxed{
    \vec{\nabla}_r \times \vec{A}_n
    &=
    \frac{-1}{2 B_0^2}
    \hat{e}_{B_0}}
\end{align}
Putting this result into Equation~\eqref{eq:gamn} we arrive at our final answer, where $d\vec{S} = B_0^2 d \Omega \hat{e}_{B_0}$
\begin{align}
    \gamma_n
    &=
    \int_s
    (
    \vec{\nabla}_r \times \vec{A}_n
    )
    \cdot
    d \vec{S}
    \\
    &=
    \int_s
    \left(
    \frac{-1}{2 B_0^2}
    \hat{e}_{B_0}
    \right)
    \cdot
    B_0^2 d \Omega \hat{e}_{B_0}
    \\
    &=
    \frac{-1}{2}
    \int d \Omega
    \\
    &=
    \frac{-1}{2}
    \int_0^\alpha \sin{\alpha} d \alpha \int_0^{2\pi} d \psi
    \\
    &= (1 - \cos{\alpha})\pi
    \\
    \Aboxed{
    \gamma_n
    &= 
    -\frac{1}{2} \Omega}
\end{align}

\problem{Fermion or Boson}
Explain how according to the algebra of quantum mechanical angular momentum, particles must exist in the form of either a fermion (with half-integral spin) or a boson (with integral spin), and cannot have any other values (rational or irrational) for their intrinsic spin such as $s = 2/3$, $7/5$, $\pi$, ..., etc.
\solution
If particles were allowed to have rational or irrational values for their intrinsic spin, the ladder operators can't lower $m$ all the way down to $m_\text{min}$ in integer steps without going beyond $m_min$ or never quite reaching it.  

Looking at our ladder operators derived in the notes[I-3]:
\begin{align}
	J_+ \ket{j m} &= c_{jm}^+ \ket{jm+1}
	\\
	J_- \ket{j m} &= c_{jm}^- \ket{jm-1}
\end{align}
and solving the eigenvalue problem to find the bounds on $m$ we find
\begin{align}
	m_\text{min} &= - m_\text{max}
	\\
	m_\text{min} &= -j
\end{align}
we are left to work on the limits of $j$, where the question lies.  Why does $j$ have to take a $1/2$ integer value?  The answer lies in the requirements for the minimum and maximum values of having to equal $|j|$, which depends on $m_\text{min}$.  If we run our lowering operator all the way to the bottom, we quickly find that for all of these conditions to be met, $j$ has to take on these positive $1/2$ integer values.
\begin{align}
	J_- \ket{j m_\text{max}} 
	&=
	J_- \ket{j j} 
	= 
	c^- \ket{j(j-1)}
	\\
	J_- \ket{j (j-1)} 
	&=
	c^- \ket{j(j-2)}
	\\
	&\vdots
	\\
	J_- \ket{j (j_\text{min}+1)} 
	&=
	c^- \ket{j(j_\text{min})}
	\\
	J_- \ket{j (j_\text{min})} 
	&=
	0
\end{align}
So if we run this process with some value with $j$, subtracting 1 from $m_\text{max}$ each time, we find that we will only get to $(j_\text{min})$ if $j = \frac{1}{2}n$ where $n$ is a positive integer.  For example, if we try with $j = 1.1$, the lowest we can get is only $m_\text{min} = -0.9$ such that $m_\text{max} \neq -m_\text{min}$.  Now that is some dense notation.


\problem{The W-Boson}
The $W$-Boson is one of the fundamental mediator of quanta in the unified electroweak field theory.  The $W$ was discovered in CERN in 1983 and is known to have spin equal to 1.  Derive explicit matrix representations for the spin operator of the $W$ particle: i.e., let $\bar{S}= \hbar \bar{\Sigma}$, find all the component matrices of $\bar{\Sigma}$, which include: $\Sigma_+$,$\Sigma_-$,$\Sigma_x$,$\Sigma_y$,$\Sigma_z$,$\Sigma^2.$
\solution
Lets assume here that we have a free particle such that the orbital angular momentum is zero: $\hat{L} = \cancelto{0}{\hat{J}} + \hat{S}$. This leaves us to find $\hat{S}$ for a spin 1 particle, which means we can follow the notes on page [I-8].  First, lets get our tools out, including our raising and lowering operators.
\begin{align}
    S_+\ket{jm}
    &=
    c_{jm}^+\ket{j(m+1)}
    &
    c_{jm}^+
    &=
    \hbar\sqrt{(j-m)(j+m+1)}
    \\
    S_+\ket{jm}
    &=
    c_{jm}^-\ket{j(m-1)}
    &
    c_{jm}^-
    &=
    \hbar\sqrt{(j+m)(j-m+1)}
    \\
    \label{eq:sx}
    S_x\ket{jm}
    &=
    \frac{1}{2}(S_+ + S_-)\ket{jm}
    \\
    \label{eq:sy}
    S_y\ket{jm}
    &=
    \frac{i}{2}(S_- - S_+)\ket{jm}
    \\
    \label{eq:sz}
    S_z\ket{jm}
    &=
    \hbar m\ket{jm}
    \\
    \label{eq:s2}
    S^2\ket{jm}
    &=
    \hbar^2 j (j+1)\ket{jm}
\end{align}
Lets go ahead and calculate our $c_{jm}^\pm$ values.  For spin $1$ particles, $j=1$, $m_\text{min}=-j$ and $m_\text{max}=j$, with integer steps in between.
\begin{align}
    c_{11}^+ &= 0   &    c_{10}^+ &= \sqrt{2}\hbar  &   c_{1(-1)}^+ &= \sqrt{2}\hbar
    \\
    c_{11}^- &= \sqrt{2}\hbar   &    c_{10}^- &= \sqrt{2}\hbar  &   c_{1(-1)}^+ &= 0
\end{align}
Our matrix elements take the form of:
\begin{equation}
    \bra{j'm'}S_\pm\ket{jm} = c_{jm}^\pm \delta_{jj'}\delta_{m'(m\pm1)}
\end{equation}
$j = j'$ for all elements however, so we can skip writing it in our matrix elements.
\begin{align}
    S_{+}\ket{j m}
    &=
    \begin{pmatrix}
        c_{11}^+ \delta_{12}    &   c_{10}^+ \delta_{11}    &   c_{1(-1)}^+ \delta_{10}    \\
        c_{11}^+ \delta_{02}    &   c_{10}^+ \delta_{01}    &   c_{1(-1)}^+ \delta_{00}    \\
        c_{11}^+ \delta_{(-1)2}    &   c_{10}^+ \delta_{(-1)1}    &   c_{1(-1)}^+ \delta_{(-1)0}
    \end{pmatrix}
    \ket{j m}
    \\
    &=
    \begin{pmatrix}
        0   &   \sqrt{2}\hbar   &   0   \\
        0   &   0   &   \sqrt{2}\hbar   \\
        0   &   0   &   0   
    \end{pmatrix}
    \ket{j m}
    \\
    S_{+}
    &=
    \sqrt{2}\hbar
    \begin{pmatrix}
        0   &   1   &   0   \\
        0   &   0   &   1   \\
        0   &   0   &   0   
    \end{pmatrix}
    \\
    \Aboxed{
    \Sigma_+
    &=
    \sqrt{2}
    {\begin{pmatrix}
        0   &   1   &   0   \\
        0   &   0   &   1   \\
        0   &   0   &   0   
    \end{pmatrix}}
    }
\end{align}
Same procedure for $S_{+}$:
\begin{align}
    S_{-}\ket{j m}
    &=
    \begin{pmatrix}
        c_{11}^- \delta_{10}    &   c_{10}^- \delta_{1(-1)}    &   c_{1(-1)}^- \delta_{1(-2)}    \\
        c_{11}^- \delta_{00}    &   c_{10}^- \delta_{0(-1)}    &   c_{1(-1)}^- \delta_{0(-2)}    \\
        c_{11}^- \delta_{(-1)0}    &   c_{10}^- \delta_{(-1)(-1)}    &   c_{1(-1)}^- \delta_{(-1)(-2)}
    \end{pmatrix}
    \ket{j m}
    \\
    &=
    \begin{pmatrix}
        0   &   0   &   0   \\
        \sqrt{2}\hbar   &   0   &   0   \\
        0   &   \sqrt{2}\hbar   &   0   
    \end{pmatrix}
    \ket{j m}
    \\
    S_{-}
    &=
    \sqrt{2}\hbar
    \begin{pmatrix}
        0   &   0   &   0   \\
        1   &   0   &   0   \\
        0   &   1   &   0   
    \end{pmatrix}
    \\
    \Aboxed{
    \Sigma_-
    &=
    \sqrt{2}
    {\begin{pmatrix}
        0   &   0   &   0   \\
        1   &   0   &   0   \\
        0   &   1   &   0   
    \end{pmatrix}}
    }
\end{align}
Solving for $S_x$ using Equation~\eqref{eq:sx}:
\begin{align}
    S_{x}\ket{j m}
    &=
    \frac{1}{2}(S_+ + S_-)\ket{j m}
    \\
    &=
    \frac{1}{2}
    \left(
    \begin{pmatrix}
        0   &   \sqrt{2}\hbar   &   0   \\
        0   &   0   &   \sqrt{2}\hbar   \\
        0   &   0   &   0   
    \end{pmatrix}
    +
    \begin{pmatrix}
        0   &   0   &   0   \\
        \sqrt{2}\hbar   &   0   &   0   \\
        0   &   \sqrt{2}\hbar   &   0   
    \end{pmatrix}
    \right)\ket{j m}
    \\
    &=
    \frac{\hbar}{\sqrt{2}}
    {\begin{pmatrix}
        0   &   1   &   0   \\
        1   &   0   &   1   \\
        0   &   1   &   0   
    \end{pmatrix}}
    \ket{j m}
    \\
    S_{x}
    &=
    \frac{\hbar}{\sqrt{2}}
    {\begin{pmatrix}
        0   &   1   &   0   \\
        1   &   0   &   1   \\
        0   &   1   &   0   
    \end{pmatrix}}
    \\
    \Aboxed{
    \Sigma_{x}
    &=
    \frac{1}{\sqrt{2}}
    {\begin{pmatrix}
        0   &   1   &   0   \\
        1   &   0   &   1   \\
        0   &   1   &   0   
    \end{pmatrix}}}
\end{align}
Solving for $S_y$ using Equation~\eqref{eq:sy}:
\begin{align}
    S_{y}\ket{j m}
    &=
    \frac{i}{2}(S_- - S_+)\ket{j m}
    \\
    &=
    \frac{i}{2}
    \left(
    \begin{pmatrix}
        0   &   0   &   0   \\
        \sqrt{2}\hbar   &   0   &   0   \\
        0   &   \sqrt{2}\hbar   &   0   
    \end{pmatrix}
    -
    \begin{pmatrix}
        0   &   \sqrt{2}\hbar   &   0   \\
        0   &   0   &   \sqrt{2}\hbar   \\
        0   &   0   &   0   
    \end{pmatrix}
    \right)\ket{j m}
    \\
    &=
    \frac{i \hbar}{\sqrt{2}}
    {\begin{pmatrix}
        0   &   -1   &   0   \\
        1   &   0   &   -1   \\
        0   &   1   &   0   
    \end{pmatrix}}
    \ket{j m}
    \\
    S_{y}
    &=
    \frac{i\hbar}{\sqrt{2}}
    {\begin{pmatrix}
        0   &   -1   &   0   \\
        1   &   0   &   -1   \\
        0   &   1   &   0   
    \end{pmatrix}}
    \\
    \Aboxed{
    \Sigma_{y}
    &=
    \frac{i}{\sqrt{2}}
    {\begin{pmatrix}
        0   &   -1   &   0   \\
        1   &   0   &   -1   \\
        0   &   1   &   0   
    \end{pmatrix}}}
\end{align}
Solving for $S_z$ using Equation~\eqref{eq:sz}:
\begin{align}
    S_{z}\ket{j m}
    &=
    \hbar m\ket{j m}
    \\
    &=
    \begin{pmatrix}
        \hbar   &   0   &   0   \\
        0   &   0   &   0   \\
        0   &   0   &   -\hbar
    \end{pmatrix}
    \ket{j m}
    \\
     S_{z}
     &=
     \hbar 
     {\begin{pmatrix}
         1   &   0   &   0   \\
         0   &   0   &   0   \\
         0   &   0   &   -1
     \end{pmatrix}}
     \\
     \Aboxed{
     \Sigma_{z}
     &=
     {\begin{pmatrix}
          1   &   0   &   0   \\
          0   &   0   &   0   \\
          0   &   0   &   -1
      \end{pmatrix}}}
\end{align}
Solving for $S^2$ using Equation~\eqref{eq:s2}:
\begin{align}
    S^{2}\ket{j m}
    &=
    \hbar^2 j(j+1)\ket{j m}
    \\
    &=
    \begin{pmatrix}
        2\hbar^2   &   0   &   0   \\
        0   &   2\hbar^2   &   0   \\
        0   &   0   &   2\hbar^2
    \end{pmatrix}
    \ket{j m}
    \\
     S_{z}
     &=
     2\hbar^2 
     {\begin{pmatrix}
         1   &   0   &   0   \\
         0   &   1   &   0   \\
         0   &   0   &   1
     \end{pmatrix}}
     \\
     \Aboxed{
     \Sigma^{2}
     &=
     2
     {\begin{pmatrix}
          1   &   0   &   0   \\
          0   &   1   &   0   \\
          0   &   0   &   1
      \end{pmatrix}} = 2 \mathbf{I}}
\end{align}




\bibliography{hw2-619} 
\bibliographystyle{plain} \nocite{*}

\end{document}