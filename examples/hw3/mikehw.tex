\documentclass{article}
\everymath{\displaystyle}
\usepackage{color}
\usepackage{bm}
\usepackage{amsmath}
\usepackage{setspace}
\usepackage{verbatim}
\usepackage{graphicx}
\usepackage[lmargin=2.5cm, rmargin=2.5cm,tmargin=3cm,bmargin=2.5cm]{geometry}
\usepackage{hyperref}
\hypersetup{
colorlinks=true,
linkcolor=blue,
urlcolor=blue
}

% \TISCH{m}{x}{V}{E}{\Psi}
\def\TISCH#1#2#3#4#5{-\frac{\hbar}{2{#1}}\frac{d^2{#5}}{d{#2}^2}+{#3}{#5}={#4}{#5}}
%
\def\ket#1{|#1\rangle}
\def\bra#1{\langle#1|}
\def\braket#1#2{\langle#1|#2\rangle}
\def\Vec#1{\mathbf{\boldsymbol{#1}}}
\def\Hat#1{\hat{\boldsymbol{#1}}}
\def\Sin#1{\sin\left(#1\right)}
\def\SinSq#1{\sin^2\left(#1\right)}
\def\Cos#1{\cos\left(#1\right)}
\def\CosSq#1{\cos^2\left(#1\right)}
\def\Sinh#1{\sinh\left(#1\right)}
\def\Cosh#1{\cosh\left(#1\right)}
\def\Exp#1{\exp\left(#1\right)}
\def\sinser#1#2{\sum_{n=1}^{\infty}\Sin{\frac{n\pi #1}{#2}}}
\def\slode#1#2#3#4#5{\frac{d}{d#2}\left(#3\frac{d#1}{d#2}\right)+#4#1+\lambda #5#1=0}
\def\slser#1{\sum_{n=1}^{\infty}a_n\phi_n(#1)e^{-\lambda t}}
\def\slan#1#2#3#4{a_n=\frac{\int_{#3}^{#4}f(#1)\phi_n(#1)#2\,d#1}{\int_{#3}^{#4}\phi_n^2(#1)#2\,d#1}}
\def\slconst#1#2#3#4#5#6{#5=\frac{\int_{#3}^{#4}#1\phi_n(#6)#2\,d#6}{\int_{#3}^{#4}\phi_n^2(#6)#2\,d#6}}
\def\eean0#1#2#3#4#5{\frac{\int_0^{#1}{#2}{#3}\,{#5}\;d{#4}} {\int_0^{#1}\left[{#3}\right]^2{#5}\;d{#4}}}
\def\eeant#1#2#3#4#5#6#7{e^{-#4#5t}\left(a_n(0) + \int_0^t \frac{\int_0^{#1}{#2}{#3}\,{#7}\;d{#6}}{\int_0^{#1}\left[{#3}\right]^2{#7}\;d{#6}} e^{#4#5\tau} d\tau\right)}

\begin{document}
\setlength{\parindent}{0pt}
\linespread{1.25} \selectfont
\fontsize{12}{12}\selectfont

{\bf
\begin{center}
PH 617, Problem Set \#2
\vspace{12pt}\\
Mike Witt\\
\end{center}
}
\bigskip

{\bf Problem 1:}
\\\\
First I'll set up the time derivatives of $\psi$ and $\psi^*$ with the
complex potential. This comes from just dividing the TDSE by $i\hbar$.
\begin{quote}
$
\frac{\partial\psi}{\partial t}
    = \frac{i\hbar}{2m} \frac{\partial^2\psi}{\partial x^2}
    - \frac{i}{\hbar}(V_R + iV_I)\psi, \;\;
\frac{\partial\psi^*}{\partial t}
    = -\frac{i\hbar}{2m} \frac{\partial^2\psi^*}{\partial x^2}
    + \frac{i}{\hbar}(V_R - iV_I)\psi^*
$
\end{quote}
Now I'll calculate the time derivative of the probability density.
I'm going use $\rho'$ in order to distinguish this density from $\rho$
which we have already calculated in class from a real potential.
\begin{quote}
$
\frac{\partial\rho'}{\partial t}
    = \frac{\partial}{\partial t}(\psi^*\psi)
    \vspace{12pt}\\ \hspace*{18pt}
    = -\frac{i\hbar}{2m}\frac{\partial^2\psi^*}{\partial x^2}\psi
      + \frac{i}{\hbar}(V_R-iV_I)\psi^*\psi
      +\frac{i\hbar}{2m}\frac{\partial^2\psi}{\partial x^2}\psi^*
      - \frac{i}{\hbar}(V_R+iV_I)\psi\psi^*
    \vspace{12pt}\\ \hspace*{18pt}
    = -\frac{i\hbar}{2m}\frac{\partial^2\psi^*}{\partial x^2}\psi
      + \frac{1}{\hbar}(iV_R+V_I)\psi^*\psi
      +\frac{i\hbar}{2m}\frac{\partial^2\psi}{\partial x^2}\psi^*
      + \frac{1}{\hbar}(-iV_R+V_I)\psi\psi^*
    \vspace{12pt}\\ \hspace*{18pt}
    = \frac{i\hbar}{2m}\left(\frac{\partial^2\psi}{\partial x^2}\psi^*
        - \frac{\partial^2\psi^*}{\partial x^2}\psi\right)
        + \frac{1}{\hbar}V_R\psi^*\psi
        - \frac{1}{\hbar}V_R\psi^*\psi
        + \frac{1}{\hbar}V_I\psi^*\psi
        + \frac{1}{\hbar}V_I\psi^*\psi
    \vspace{12pt}\\ \hspace*{18pt}
    = \frac{i\hbar}{2m}\left(\frac{\partial^2\psi}{\partial x^2}\psi^*
        - \frac{\partial^2\psi^*}{\partial x^2}\psi\right)
        + \frac{2}{\hbar}V_I\psi^*\psi
    \vspace{12pt}\\ \hspace*{18pt}
    = \frac{\partial\rho}{\partial t} + \frac{2}{\hbar}V_I\psi^*\psi
$
\end{quote}
So the new probability density $\rho'$ appears to be equal to the
probability density $\rho$ which was previously calculated from a real
potential, but with the added term of:
$(2/\hbar)V_I\psi^*\psi$. The integration is easy, because we already
know that total time change of $\rho$ is zero, and the inner product
$\langle \psi^*|\psi\rangle$ is one.
\begin{quote}
$
   \int_{-\infty}^{\infty}
    \left(
        \frac{\partial\rho}{\partial t} + \frac{2}{\hbar}V_I\psi^*\psi
    \right)dx
   = \int_{-\infty}^{\infty} \frac{\partial\rho}{\partial t}\,dx
   + \frac{2}{\hbar}V_I\int_{-\infty}^{\infty} \psi^*\psi\,dx
    = \frac{2}{\hbar}V_I
$
\end{quote}
If I'm understanding this all correctly, then in answer to the question:
if we want the total time change to be negative, then we must start out
with a potential that has a negative imaginary part.

\pagebreak
{\bf Problem 2(a):}
\\\\
$\psi = Ae^{ikx} + Be^{-ikx}, \;\;
        \frac{\partial\psi}{\partial x} = ikAe^{ikx} - ikBe^{-ikx}
    \vspace{6pt}\\
    \psi^* = A^*e^{-ikx} + B^*e^{ikx}, \;\;
        \frac{\partial\psi^*}{\partial x} = -ikA^*e^{-ikx} + ikB^*e^{ikx}
\vspace{12pt}\\
    J = \frac{i\hbar}{2m}\left[
        \psi\frac{d\psi^*}{dx} - \psi^*\frac{d\psi}{dx} \right]
\vspace{12pt}\\ \hspace*{8pt}
    = \frac{i\hbar}{2m}\left[
      \left(Ae^{ikx}+Be^{-ikx}\right)\left(-ikA^*e^{-ikx}+ikB^*e^{ikx}\right) 
      -\left(A^*e^{-ikx}+B^*e^{ikx}\right)\left(ikAe^{ikx}-ikBe^{-ikx}\right)
    \right]
\vspace{12pt}\\ \hspace*{8pt}
    = \frac{i\hbar}{2m}\left[
        -ikAA^* + ikABe^{2ikx} - ikABe^{-2ikx} + ikBB^*
        -ikAA^* +ikABe^{-2ikx} - ikABe^{2ikx} + ikBB^*
    \right]
\vspace{12pt}\\ \hspace*{8pt}
    = \frac{i\hbar}{2m}\left[ -2ikAA^* +2ikBB^* \right]
\vspace{12pt}\\ \hspace*{8pt}
    = (2ik)\frac{i\hbar}{2m}\left[ -AA^* +BB^* \right]
\vspace{12pt}\\ \hspace*{8pt}
    = \frac{k\hbar}{m}\left(\,|A|^2 - |B|^2\,\right)
$
\\\\\\
{\bf Problem 2(b):}
\\\\
$
\psi = u(x)e^{ikx},\;\; \psi^* = u(x)e^{-ikx}
\vspace{12pt}\\
\frac{d\psi}{dx} = u'e^{ikx}+ikue^{ikx},\;\;
\frac{d\psi^*}{dx} = u'e^{-ikx}-ikue^{-ikx}
\vspace{12pt}\\
  J = \frac{i\hbar}{2m}\left[
      ue^{ikx}\left(u'e^{-ikx}-ikue^{-ikx}\right) 
      -u(x)e^{-ikx}\left(u'e^{ikx}+ikue^{ikx}\right)
    \right]
\vspace{12pt}\\ \hspace*{8pt}
  = \frac{i\hbar}{2m}\left[ uu' - iku^2 - uu' - iku^2 \right]
\vspace{12pt}\\ \hspace*{8pt}
  = \frac{i\hbar}{2m}\left[-2iku^2 \right]
\vspace{12pt}\\ \hspace*{8pt}
  = \frac{\hbar k}{m}\left[u(x)\right]^2
$

\pagebreak
{\bf Problem 3:}
\\\\
I'll start with the expectation value for position, expressed in the
position basis:
\begin{quote}
$
    \langle x \rangle =
    \int_{-\infty}^{\infty} \psi^*\, x\, \psi\, dx
$
\end{quote}
Then replace $\psi$ with its Fourier transform:
\begin{quote}
$
    \langle x \rangle =
    \int_{-\infty}^{\infty} \psi^* \, x\,
    \left(\frac{1}{\sqrt{2\pi\hbar}}
    \int_{-\infty}^{\infty}\phi\, e^{i p_x x/\hbar}\,dp_x\right)
    dx
$
\end{quote}
Now, I can swap the order of the integrations as long as all the
functions of $x$ and $p_x$ stay within their respective integrals:
\begin{quote}
$
    \langle x \rangle =
    \int_{-\infty}^{\infty} \phi
    \left(\frac{1}{\sqrt{2\pi\hbar}}
    \int_{-\infty}^{\infty} \psi^*\,x\,e^{i p_x x/\hbar}\,dx\right)
    dp_x
    \;\;\; (1)
$
\end{quote}
I can take the conjugate of $\phi$ and its derivative by operating
directly on the Fourier transform:
\begin{quote}
$
    \frac{d\phi^*}{dp_x} =
    \frac{1}{\sqrt{2\pi\hbar}} \int_{-\infty}^{\infty} 
    \psi^* \frac{d}{dp_x}\left(e^{i p_x x/\hbar}\right)\,dx
    = \frac{ix}{\hbar}\frac{1}{\sqrt{2\pi\hbar}} \int_{-\infty}^{\infty} 
    \psi^* e^{i p_x x/\hbar}\,dx
$
\end{quote}
Therefore:
\begin{quote}
$
  \frac{1}{\sqrt{2\pi\hbar}} \int_{-\infty}^{\infty} 
    \psi^* x\,e^{i p_x x/\hbar}\,dx
   = -i\hbar\frac{d\phi^*}{dp_x}
$
\end{quote}
Substituting this value back into (1) gives:
\begin{quote}
$
    \langle x \rangle =
    \int_{-\infty}^{\infty}\phi\left( -i\hbar\frac{d\phi^*}{dp_x}\right) dp_x
    = -i\hbar\int_{-\infty}^{\infty}\phi\,\frac{d\phi^*}{dp_x}\,dp_x
$
\end{quote}
Which by the product rule (integration by parts) is:
\begin{quote}
$
    \langle x \rangle 
    = -i\hbar\int_{-\infty}^{\infty} 
        \left(
        \frac{d\phi}{dp_x}(\phi\phi^*)
        - \phi^*\,\frac{d\phi}{dp_x}\right) dp_x
    \vspace{12pt}\\ \hspace*{15pt}
    = -i\hbar\left(\phi\phi^*\bigg|_{-\infty}^0
        +\phi\phi^*\bigg|_0^{\infty} 
        - \int_{-\infty}^{\infty}\phi^*\,\frac{d\phi}{dp_x} dp_x
        \right)
    \vspace{12pt}\\ \hspace*{15pt}
    = -i\hbar\left(
        +\phi(0)\phi^*(0) - 0 + 0 - \phi(0)\phi^*(0)
        - \int_{-\infty}^{\infty}\phi^*\,\frac{d\phi}{dp_x} dp_x
        \right)
    \vspace{12pt}\\ \hspace*{15pt}
        = i\hbar\int_{-\infty}^{\infty}\phi^*\,\frac{d\phi}{dp_x} dp_x
        = \int_{-\infty}^{\infty}
        \phi^*
        \left(i\hbar\frac{d}{dp_k}\right)
        \phi\,dp_x
$
\end{quote}
Which expresses the expectation value in the momentum basis, utilizing
the operator: 
\begin{quote}
    $\hat{x} = i\hbar\frac{d}{dp_k}$.
\end{quote}

\pagebreak
{\bf Problem 4:}
\\\\
I believe what we have here is: 
$e^{i\hat{p}_xa/\hbar}\,\hat{x}\,e^{-i\hat{p}_xa/\hbar}$\\
Since we are dealing with operators, we can't simply add the exponents.
\\\\
I'm going to use formula (2.117) from Zettili page 97 to do the Taylor
series expansion using the commutator:
\begin{quote}
$
    e^{i\hat{p}_xa/\hbar}\,\hat{x}\,e^{-i\hat{p}_xa/\hbar}
    = \hat{x} +\left[i\hat{p}_xa/\hbar,\;\hat{x}\right]
    + \left[i\hat{p}_xa/\hbar,\;\left[i\hat{p}_xa/\hbar,\;\hat{x}\right]\right]
    + ...
$
\end{quote}
the first term is: $\hat{x}$
\vspace{6pt}\\
the 2nd term is: $\left[i\hat{p}_xa/\hbar,\;\hat{x}\right]
    = (i/\hbar)a\left[\hat{p}_x,\;\hat{x}\right]
    = (i/\hbar)a(-i\hbar) = a
$
\vspace{6pt}\\
the 3rd term is:
 $\left[i\hat{p}_xa/\hbar,\;\left[i\hat{p}_xa/\hbar,\;\hat{x}\right]\right]
    = \left[i\hat{p}_xa/\hbar,\;a\right]=0\;$ (all opers commute with
    constants)
\vspace{6pt}\\
and all the rest of the terms are zero, since they include the (nested)
commutation with zero.\\\\
So we have: $\hat{x}+a = x+a$ (in the position basis).
\\\\
The problem said to "use the results of problem 3" and I guess I didn't
do that. So maybe I did it the wrong way?


\pagebreak
{\bf Problem 5(a):}
\\\\
(i) Assuming that $A$ is Hermitian all of its eigenvalues will be real, and
a real number squared is positive, therefore:
\begin{quote}
$
    \langle A^2 \rangle = \langle \psi| A^2 |\psi\rangle
    = \langle \psi| A (A|\psi\rangle)
    = \langle \psi| A (\lambda|\psi\rangle)
    = \langle \psi| \lambda^2|\psi\rangle
    = \lambda^2 \langle \psi| \psi\rangle
    \vspace{6pt}\\
    $where $\lambda^2$ is real and positive
\end{quote}
Assuming that $\psi$ is normalized, the result will just be $\lambda^2$.
But, in any event, the product of $\psi$ and its conjugate will be real
and positive and so will the final result. Therefore $\langle A^2 \rangle
\ge 0$.
\\\\
(ii) Let $B=e^{iA}$. $B^\dagger=\left(e^{iA}\right)^\dagger = e^{-iA^\dagger}$ 
    which, since $A$ is Hermitian, $=e^{-iA}$. This makes $BB^\dagger
    = e^{iA-iA}$. Since $A$ commutes with itself, this really is $1$ even
    though $A$ is an operator. Therefore $B$ is unitary.
\\\\\\
{\bf Problem 5(b):}
\\\\
If $\psi$ is normalized, then $\langle \psi|\psi\rangle = 1$. If
$U$ is unitary and $\phi=U|\psi\rangle$ then:
\begin{quote}
$
    \langle \phi|\phi\rangle
    = \langle U\psi|U\psi\rangle
    \vspace{6pt}\\ \hspace*{27pt}
    = \langle \psi|U^\dagger U\psi\rangle
    \vspace{6pt}\\ \hspace*{27pt}
    = \langle \psi|I\psi\rangle
$
\end{quote}
Just as an example, here's how it works out with $U=e^{ix}$:
\begin{quote}
$
    \int \psi^*(x) \psi(x)\,dx = 1,\;\; \phi(x) = e^{ix}\psi(x)
    \vspace{6pt}\\
    \int \phi^*(x) \phi(x)\,dx
    = \int \left[e^{ix}\psi(x)\right]^* \left[e^{ix}\psi(x)\right]\,dx
    \vspace{6pt}\\ \hspace*{78pt}
    = \int \psi^*(x)e^{-ix}e^{ix}\psi(x)\,dx
    \vspace{6pt}\\ \hspace*{78pt}
    = \int \psi^*(x)\,(1)\,\psi(x)\,dx
$
\end{quote}
For an operator that "sticks" to a function (like a derivative operator)
it still works, but requires using the boundary conditions for the specific
space to see all the details play out. For example using the product rule
(integration by parts) as was done in problem 3.

\pagebreak
{\bf Problem 6:} 
\\\\
We are dealing with bound states, so the energy levels $E_n$ are discrete.
We are given two different basis states: $\psi_1$ and $\psi_2$.
\\\\
(a) Show that the following expression is constant
\begin{quote}
    $\psi_1' \psi_2 - \psi_1 \psi_2'\;\;(1)$
\end{quote}
First solve the TDSE for $E_n$
\begin{quote}
$
    -\frac{\hbar^2}{2m}\frac{\partial^2\psi}{\partial x^2}+V(x)\psi
        = E_n\psi
  \vspace{6pt}\\
    E_n =
    -\frac{1}{\psi}\frac{\hbar^2}{2m}\frac{\partial^2\psi}{\partial x^2}+V(x)
    \;\;\;$(just divide through by $\psi$)$
$
\end{quote}
Since both $\psi_1$ and $\psi_2$ have the {\it same} $E_n$
\begin{quote}
$
    -\frac{1}{\psi_1}\frac{\hbar^2}{2m}\frac{\partial^2\psi_1}{\partial x^2}
    +V(x)
    =
    -\frac{1}{\psi_2}\frac{\hbar^2}{2m}\frac{\partial^2\psi_2}{\partial x^2}
    +V(x)
  \vspace{10pt}\\
    -\frac{1}{\psi_1}\frac{\hbar^2}{2m}\frac{\partial^2\psi_1}{\partial x^2}
    =
    -\frac{1}{\psi_2}\frac{\hbar^2}{2m}\frac{\partial^2\psi_2}{\partial x^2}
    \;\;\; $(potentials cancel)$
  \vspace{10pt}\\
    \frac{1}{\psi_1}\frac{\partial^2\psi_1}{\partial x^2}
    =
    \frac{1}{\psi_2}\frac{\partial^2\psi_2}{\partial x^2}
    \;\;\;$(divide through by all the constants)$
  \vspace{10pt}\\
    \psi_2\frac{\partial^2\psi_1}{\partial x^2}
    =
    \psi_1\frac{\partial^2\psi_2}{\partial x^2}
    \;\;\;$(multiply through by $\psi_1\psi_2$)$
  \vspace{10pt}\\
    $(and switch notation :-)$
  \vspace{10pt}\\
    \psi_2\psi_1'' = \psi_1\psi_2'' \;\;(2)
$
\end{quote}
Now, take the derivative of (1)
\begin{quote}
$
    \frac{\partial}{\partial x}(\psi_1' \psi_2 - \psi_1 \psi_2')
    = \psi_1''\psi_2+\psi_1'\psi_2' - \psi_1'\psi_2' - \psi_1\psi_2''
    = \psi_1''\psi_2 - \psi_1\psi_2''
$
\end{quote}
But from (2) we see that the derivative of (1) is zero, which means
that (1) must be constant.

\pagebreak
%%%%%%%%%%%%%%%%%%%%%%%%%%%%%%%%%%%%%%%%%%%%%%%%%%%%%%%%%%%%%%%%%%%%%%%%%%
\begin{comment}
(b) We are talking about {\it bound states} here. So it seems like I can't
    just say that $\psi_1$ and $\psi_2$ must go to zero at the endpoints
    in order to be square integrable. For example, they might be defined
    on, say, the interval $[0,1]$ in which case they would be square
    integrable as long as they are everywhere finite.
    \\\\
    However, I suppose the concept of
    a "bound" state implies that the wave function is restricted to within
    the boundary, and hence must go to zero at the boundary.
    So assuming that $\psi_1$ and $\psi_2$ {\it do} 
    go to zero at the endpoints, then the only way that either of the
    terms of (1) would not go to zero would be if one of the derivatives
    increased without bound. That clearly couldn't happen,
    {\it if the function is continuous}, or else the
    function itself would never settle down to zero. {\it I suppose, if
    you were to take an infinite well, and say that the initial condition
    was constant inside the well but zero at the boundary, then you would
    have an infinite derivative. But the price would be a discontinuous
    wave function.}
\\\\
    In any event, for a physically realizable system it appears that
    the argument works.
\\\\
\end{comment}
%%%%%%%%%%%%%%%%%%%%%%%%%%%%%%%%%%%%%%%%%%%%%%%%%%%%%%%%%%%%%%%%%%%%%%%%%%
(b) The problem says "use the fact that bound state wave functions must
    be square integrable" but I don't really understand this. Since bound
    states might be on a finite interval, square integrability only means
    that the wave functions must be "finite." For example, the boundary
    conditions might be "periodic" and so the function would not have to
    go to zero at the end points.
\\\\
    Perhaps square integrability is just a "stand in" for saying that
    the function is in fact an element of the appropriate Hilbert space.
    So it must be square integrable {\it and} it must have some
    appropriate boundary conditions. Once the specific boundary conditions
    are specified, then presumably the problem could be worked out
    using integration by parts. But I don't know how to do it without
    the specific BCs.
\\\\
    Maybe the idea is something like this ... A Sturm-Liouville
    eigenvalue problem can be written as:
    \begin{quote}
    $
    \frac{d}{dx} \left(p\frac{d\psi}{dx}\right) +q\psi+ \lambda \sigma\psi=0
    $
    \end{quote}
    The TISE can be put into that form:
    \begin{quote}
    $
    -\frac{\hbar^2}{2m}\frac{d^2}{dx^2}\psi + V(x)\psi = E\psi
    \vspace{12pt}\\
    \frac{d}{dx} \left(-\frac{\hbar^2}{2m}\frac{d\psi}{dx}\right)
        +V(x)\psi -E \psi=0
    \vspace{12pt}\\
    \frac{d}{dx} \left(\frac{\hbar^2}{2m}\frac{d\psi}{dx}\right)
        -V(x)\psi+ E \psi=0
    $
    \vspace{12pt}\\
    Where:
        \begin{quote}
        $
        p = \frac{\hbar^2}{2m},\; $(real, positive)$
        \vspace{6pt}\\
        q = V(x),\; $(real)$
        \vspace{6pt}\\
        \lambda = E
        \vspace{6pt}\\
        \sigma = 1\; $(real, positive)$
        $
        \end{quote}
    \end{quote}
I imagine that for any Hilbert space we are going to be concerned with, 
$\psi$ will have {\it some}
legitimate Sturm-Liouville boundary conditions (probably $\psi=0$
or $\psi$ and its derivative periodic).
So the self-adjoint property of the Sturm-Liouville operator guarantees that
either $\psi$ or its derivative will disappear at the boundaries without
knowing the exact details of the problem.
\\\\
Somehow I suspect that this is not what you had in mind, and I suppose
that I've probably missed the point of part (b).

\pagebreak
(c) From part (b) we have that: $\psi_1'\psi_2 - \psi_1\psi_2'=0$.
    Starting from there ...
\begin{quote}
$
    \psi_1'\psi_2 = \psi_1\psi_2'
  \vspace{10pt}\\
    \frac{1}{\psi_1}\psi_1' = \frac{1}{\psi_2}\psi_2'
    \;\;$(dividing through by $\psi_1\psi_2$)$
  \vspace{10pt}\\
    \frac{1}{\psi_1}\frac{d\psi_1}{dx} = \frac{1}{\psi_2}\frac{d\psi_2}{dx}
    \;\;$(switching notation again)$
  \vspace{10pt}\\
    $Now I'm going to  multiply through by the infinitesimal $dx$. I wouldn't
    be able to do that in a math class, but I hope it's OK here.$
  \vspace{10pt}\\
    \frac{1}{\psi_1}d\psi_1 = \frac{1}{\psi_2}d\psi_2
  \vspace{10pt}\\
    \ln|\psi_1| + c_1 = \ln|\psi_2| + c_2
    \;\; $(integrate both sides)$
  \vspace{10pt}\\
    \ln|\psi_1| = \ln|\psi_2| + K
    \;\; $(lump together the constants on the right)$
  \vspace{10pt}\\
    $Now take the exponential function of both sides and work it out ...$
  \vspace{10pt}\\
    e^{\ln|\psi_1|} = e^{\ln|\psi_2| + K}
    \Rightarrow
    e^{\ln|\psi_1|} = e^Ke^{\ln|\psi_2|}
    \Rightarrow
    \psi_1 = e^K\psi_2
$
\end{quote}
Therefore $\psi_1$ is equal to $\psi_2$ times the constant $e^K$,
and when normalized they must be the same function.

\pagebreak
{\bf Problem 7:} 
\\\\
%%%%%%%%%%%%%%%%%%%%%%%%%%%%%%%%%%%%%%%%%%%%%%%%%%%%%%%%%%%%%%%%%%%%%%%%%%
\begin{comment}
(a) The first thing I'm going to do is rewrite $\psi$ (and $\psi^*$) as:
\begin{quote}
$
    \psi=\frac{1}{\sqrt{2\pi\hbar}}\int_{-\infty}^{\infty}
        \phi(p)\, \frac{\partial}{\partial p}
        \left(\frac{\hbar}{ix} e^{ipx/h}\right) dp,
    \;\;\psi^* =
    \frac{1}{\sqrt{2\pi\hbar}}\int_{-\infty}^{\infty}
        \phi^*(p)\, \frac{\partial}{\partial p}
        \left(\frac{-\hbar}{ix} e^{-ipx/h}\right) dp
$
\end{quote}
All I've done is to rewrite the exponential factor as the derivative
of its anti-derivative. The next step is to use integration by parts
to "switch" the derivative to $\phi$. This works since the boundary
condition on $\phi$ will cause the unwanted term to "disappear" during
the integration. I hope I don't have to write this all out. Exactly
the same technique was worked out in detail in Problem 3. The result is:
\begin{quote}
$
\vspace{-6pt}\\
    \psi=-\frac{1}{\sqrt{2\pi\hbar}}\int_{-\infty}^{\infty}
        \frac{d\phi}{dp}\frac{\hbar}{ix} e^{ipx/h} dp,
    \;\;\psi^* =
    \frac{1}{\sqrt{2\pi\hbar}}\int_{-\infty}^{\infty}
        \frac{d\phi^*}{dp} \frac{\hbar}{ix} e^{-ipx/h} dp
$
\end{quote}
We are given that $(\Delta x)^2=\langle x^2\rangle$ which means that:
\begin{quote}
$
    (\Delta x)^2= \langle \psi^*|x^2|\psi\rangle
    \vspace{12pt}\\
    \frac{\partial}{\partial x} \left[(\Delta x)^2\right]=
    \left(
    \frac{1}{\sqrt{2\pi\hbar}}\int_{-\infty}^{\infty}
        \frac{d\phi^*}{dp} \frac{\hbar}{ix} e^{-ipx/h} dp
    \right)
    x^2
    \left(
    -\frac{1}{\sqrt{2\pi\hbar}}\int_{-\infty}^{\infty}
        \frac{d\phi}{dp}\frac{\hbar}{ix} e^{ipx/h} dp
    \right)
    \vspace{12pt}\\
    $ The two integrations are with respect to $p$ so the two $x$ in
    the denominators of the integrations are constants. They can be
    moved out of the integrations to cancel the $x^2$ in the middle:$
    \vspace{12pt}\\
    \frac{\partial}{\partial x} \left[(\Delta x)^2\right]=
    \left(
    \frac{1}{\sqrt{2\pi\hbar}}\int_{-\infty}^{\infty}
        \frac{d\phi^*}{dp} \frac{\hbar}{i} e^{-ipx/h} dp
    \right)
    \left(
    -\frac{1}{\sqrt{2\pi\hbar}}\int_{-\infty}^{\infty}
        \frac{d\phi}{dp}\frac{\hbar}{i} e^{ipx/h} dp
    \right)
    \vspace{12pt}\\
    $Now I'll just clean things up a bit, canceling the one minus sign
    with the two $i$ and moving the $\hbar$.$
    \vspace{12pt}\\
    \frac{\partial}{\partial x} \left[(\Delta x)^2\right]=
    \left(
    \frac{1}{\sqrt{2\pi\hbar}}\int_{-\infty}^{\infty}
        \hbar\, \frac{d\phi^*}{dp}\, e^{-ipx/h} dp
    \right)
    \left(
    \frac{1}{\sqrt{2\pi\hbar}}\int_{-\infty}^{\infty}
        \hbar\, \frac{d\phi}{dp}\, e^{ipx/h} dp
    \right)
    \vspace{12pt}\\
    (\Delta x)^2=
    \int_{-\infty}^{\infty}\left(
    \frac{1}{\sqrt{2\pi\hbar}}\int_{-\infty}^{\infty}
        \hbar\, \frac{d\phi^*}{dp}\, e^{-ipx/h} dp
    \right)
    \left(
    \frac{1}{\sqrt{2\pi\hbar}}\int_{-\infty}^{\infty}
        \hbar\, \frac{d\phi}{dp}\, e^{ipx/h} dp
    \right)
    dx
\vspace{-4pt}\\
$
\end{quote}
Now I'm integrating the product of two transforms which are
conjugates of each other, and I believe I can apply a particular form of
the convolution theorem, which apparently is sometimes called
{\it Parseval's identity}. (See for example, {\it Applied Differential
Equations, Richard Haberman, Fourth Edition,} starting
at the bottom of page 467.) I believe this will allow me to essentially
multiply the two transforms together and cancel the normalization constant.
Then the exponentials will cancel each other, giving the desired result.
Unfortunately, I have not quite been able to figure
out in detail how this convolution works ... 

\pagebreak
So for now I'm going to assume the
result and hope for partial credit. Note that if this works for
$\Delta x$ then the whole process will be exactly the same for
$\Delta p$. The only difference is the sign of the exponential and that
washes out because we're also taking the conjugates.
\\\\
\end{comment}
%%%%%%%%%%%%%%%%%%%%%%%%%%%%%%%%%%%%%%%%%%%%%%%%%%%%%%%%%%%%%%%%%%%%%%%%%%
(a) 
We established (in class) that $\hat{p}$ in the position basis
is $-i\hbar\frac{\partial}{\partial x}$ and (in problem 3) that
$\hat{x}$ in the momentum basis is $i\hbar\frac{\partial}{\partial p}$.
It is given in the problem that $(\Delta x)^2=\langle x^2\rangle$ and
$(\Delta p)^2=\langle p^2\rangle$.
\\\\
It follows that:
\begin{quote}
$
    (\Delta x)^2=\langle x^2\rangle =
    \int_{-\infty}^{\infty}
        \phi^*\,\left(i\hbar\frac{\partial}{\partial p}\right)^2\phi\,dp
    \vspace{12pt}\\
    = \int_{-\infty}^{\infty}
     \phi^*\,i^2\hbar^2\left(\frac{\partial}{\partial p}\right)^2\phi\,dp
    \vspace{12pt}\\
    = -\hbar^2\int_{-\infty}^{\infty}
     \phi^*
        \frac{\partial}{\partial p}
        \frac{\partial}{\partial p}\phi\,dp
    \vspace{12pt}\\
    $Let $\hat{A}=\frac{\partial}{\partial p}$ then (in this space)
    $\hat{A}^\dagger=-\frac{\partial}{\partial p}$. We
    concluded on your blackboard the other day that this was true
    (although you seemed to disapprove of
    my calling it the "adjoint"), but I'm going with that, so we have:$
    \vspace{12pt}\\
    = -\hbar^2\int_{-\infty}^{\infty}
     \phi^*
        \frac{\partial}{\partial p}
        \frac{\partial}{\partial p}\phi\,dp
    =\langle\phi|\hat{A}\hat{A}\phi\rangle
    = \langle\hat{A}^\dagger\phi|\hat{A}\phi\rangle
    = -\hbar^2\int_{-\infty}^{\infty}
        \left(-\frac{\partial}{\partial p}\right)
        \phi^*
        \frac{\partial}{\partial p}\phi\,dp
    \vspace{12pt}\\
    = \int_{-\infty}^{\infty}
        h\frac{\partial}{\partial p}
        \phi^*
        h\frac{\partial}{\partial p}\phi\,dp
$
\end{quote}
The same process works for $(\Delta p)^2$:
\begin{quote}
$
    (\Delta p)^2=\langle p^2\rangle =
    \int_{-\infty}^{\infty}
        \psi^*\,\left(-i\hbar\frac{\partial}{\partial x}\right)^2\psi\,dp
    \vspace{12pt}\\
    =\int_{-\infty}^{\infty}
        \psi^*\,(-1)^2\,i^2\,\hbar^2
        \frac{\partial}{\partial x}
        \frac{\partial}{\partial x}
        \psi\,dp
    =-\hbar^2\int_{-\infty}^{\infty}\psi^*
        \frac{\partial}{\partial x}
        \frac{\partial}{\partial x}
        \psi\,dp
    =-\hbar^2\int_{-\infty}^{\infty}
        \left(-\frac{\partial}{\partial x}\right)
        \psi^*
        \frac{\partial}{\partial x}
        \psi\,dp
    \vspace{12pt}\\
    \int_{-\infty}^{\infty}\psi^*
        \hbar\frac{\partial}{\partial x}
        \hbar\frac{\partial}{\partial x} \psi\,dp
$
\end{quote}
\vspace{6pt}
I guess if my "calculus skills" were better I would know the right
thing to do with integrals and wouldn't have to resort to writing
the inner product in Dirac notation, but I think the argument above
is valid.

\pagebreak
(b) Given:
\begin{quote}
$
    \int_{-\infty}^{\infty}\bigg|
        \frac{x\psi}{a(\Delta x)^2} + \frac{\partial\psi}{\partial x}
        \bigg|^2 dx \ge 0 
\vspace{12pt}\\
    $Expand the modulus squared:$
\vspace{12pt}\\
    \int_{-\infty}^{\infty}
    \left(
    \frac{x\psi^*}{a(\Delta x)^2} + \frac{\partial\psi^*}{\partial x}
    \right)
    \left(
    \frac{x\psi}{a(\Delta x)^2} + \frac{\partial\psi}{\partial x}
    \right) dx \ge 0
\vspace{12pt}\\
    \int_{-\infty}^{\infty}
    \left(
        \frac{x\psi^*}{a(\Delta x)^2} \frac{x\psi}{a(\Delta x)^2}
        + \frac{x\psi^*}{a(\Delta x)^2} \frac{\partial\psi}{\partial x}
        + \frac{x\psi}{a(\Delta x)^2} \frac{\partial\psi^*}{\partial x}
        + \frac{\partial\psi^*}{\partial x} \frac{\partial\psi}{\partial x}
    \right) dx \ge 0
\vspace{12pt}\\
    $Rearrange things and integrate the first term,
    recognizing that the denominator is
    a constant, and the numerator is by definition $\langle x^2\rangle$
    which we know to be $(\Delta x)^2:
\vspace{12pt}\\
    \int_{-\infty}^{\infty} 
        \frac{1}{a(\Delta x)^2} \frac{\psi^*x^2\psi}{a(\Delta x)^2}\,dx
    + \int_{-\infty}^{\infty}
        x\left(\psi^*\frac{\partial\psi}{\partial x}
        +\psi\frac{\partial\psi^*}{\partial x}\right)\,dx
    + \int_{-\infty}^{\infty}
        \frac{\partial\psi^*}{\partial x} \frac{\partial\psi}{\partial x}\,dx
    \ge 0
\vspace{12pt}\\
        \frac{1}{a(\Delta x)^2} \frac{(\Delta x)^2}{a(\Delta x)^2}
    + \int_{-\infty}^{\infty}
        x\left(\psi^*\frac{\partial\psi}{\partial x}
        +\psi\frac{\partial\psi^*}{\partial x}\right)\,dx
    + \int_{-\infty}^{\infty}
        \frac{\partial\psi^*}{\partial x} \frac{\partial\psi}{\partial x}\,dx
    \ge 0
\vspace{12pt}\\
        \frac{1}{a^2(\Delta x)^2}
    + \int_{-\infty}^{\infty}
        x\left(\psi^*\frac{\partial\psi}{\partial x}
        +\psi\frac{\partial\psi^*}{\partial x}\right)\,dx
    + \int_{-\infty}^{\infty}
        \frac{\partial\psi^*}{\partial x} \frac{\partial\psi}{\partial x}\,dx
    \ge 0
\vspace{12pt}\\
    $Multiply the last term by $\hbar^2/\hbar^2$ to get $(\Delta p)^2:
\vspace{12pt}\\
        \frac{1}{a^2(\Delta x)^2}
    + \int_{-\infty}^{\infty}
        x\left(\psi^*\frac{\partial\psi}{\partial x}
        +\psi\frac{\partial\psi^*}{\partial x}\right)\,dx
    + \frac{1}{\hbar^2}\int_{-\infty}^{\infty}
        \hbar\frac{\partial\psi^*}{\partial x}
        \hbar\frac{\partial\psi}{\partial x}\,dx
    \ge 0
\vspace{12pt}\\
        \frac{1}{a^2(\Delta x)^2}
    + \int_{-\infty}^{\infty}
        x\left(\psi^*\frac{\partial\psi}{\partial x}
        +\psi\frac{\partial\psi^*}{\partial x}\right)\,dx
    + \frac{1}{\hbar^2}(\Delta p)^2
    \ge 0
\vspace{12pt}\\
    $Multiply through by $(\Delta x)^2$ and rearrange middle term:$
\vspace{12pt}\\
        \frac{1}{a^2}
    + (\Delta x)^2 \int_{-\infty}^{\infty}
        x\frac{\partial}{\partial x}(\psi^*\psi)\,dx
    + \frac{1}{\hbar^2}(\Delta p)^2(\Delta x)^2
    \ge 0
$
\end{quote}
{\it I'm stuck! I can't figure out what to do with the middle term.
I'm going to move on and try to work on part (c).}

\pagebreak
(c) Given:
$
    \Delta x \Delta p \ge \hbar \frac{\sqrt{a-1}}{a},\;
$
show that:
$
    \Delta x \Delta p \ge \frac{\hbar}{2}
$
\vspace{12pt}\\
Consider the factor $\sqrt{a-1}/a$.
Since $\Delta x$, $\Delta p$, and $\hbar$ are real and positive
it must be true that $a\ge 1$. Take the derivative:
\begin{quote}
$
    \frac{d}{da}\frac{\sqrt{a-1}}{a}
    \vspace{12pt}\\
    = \frac{d}{da}\left( a^{-1} (a-1)^{1/2} \right)
    \vspace{12pt}\\
    = (-1)a^{-2}(a-1)^{1/2} + a^{-1} \frac{1}{2}(a-1)^{-1/2}
    \vspace{12pt}\\
    = -\frac{\sqrt{a-1}}{a^2} + \frac{1}{2a\sqrt{a-1}}
$
\end{quote}
On the open interval $(1,+\infty)$ the
derivative has a single zero at $a=2$ so the function can only
change direction at that point. Now look at the end points:
\vspace{6pt}\\
At $a=1$ the factor is clearly zero.
\vspace{6pt}\\
As $a\rightarrow\infty$ I think it's pretty obvious that the numerator
only grows with the square root of the denominator, and so the limit
is zero. But I suppose it's best to check with l'Hopital's rule:
\begin{quote}
$
    \lim_{a\to \infty}\frac{f(a)}{g(a)} =
    \lim_{a\to \infty}\frac{f'(a)}{g'(a)}
    \vspace{12pt}\\
    = \lim_{a\to \infty} \frac{\frac{d}{da}\sqrt{a-1}} {\frac{d}{da} a}
    \vspace{12pt}\\
    = \lim_{a\to \infty}\frac{\frac{1}{2\sqrt{a-1}}}{1}
$
    \vspace{12pt}\\
    Now it's clear that the numerator goes to zero.
\end{quote}
\vspace{6pt}
So, $\frac{\sqrt{a-1}}{a}$ is zero at both endpoints and has a maximum
when $a=2$, where $\frac{\sqrt{2-1}}{2}=\frac{1}{2}$.
\\\\
The value of the uncertainty is then limited by:
$
    \Delta x \Delta p \ge \hbar\frac{1}{2}
$
and the uncertainty principle is verified.

\pagebreak
{\bf Problem 8} 
\\\\
(a) Let me start by numbering the three formulae given, so that I can
refer back to them:
\begin{quote}
$
    \vspace{-6pt}\\
    (1)\;\;\sum_{a'}
        \langle a|x|a'\rangle
        \langle a'|p|b\rangle
    - \sum_{a'}
        \langle a|p|a'\rangle
        \langle a'|x|b\rangle
    = i\hbar\langle a|b\rangle
\vspace{12pt}\\
    (2)\;\;\sum_{a'}\left(
        x_{aa'}\, p_{a'b} - p_{aa'}\, x_{a'b}
        \right) = i\hbar\delta_{ab}
\vspace{12pt}\\
    (3)\;\;\sum_{a}\sum_{a'}
        \langle a|x|a'\rangle
        \langle a'|p|a\rangle
    - \sum_{a}\sum_{a'}
        \langle a|p|a'\rangle
        \langle a'|x|a\rangle
    = i\hbar N
$
\end{quote}
Starting with formula (1), the expression $\langle a|x|a'\rangle$
simply picks out the $X$ matrix element $x_{aa'}$. So what we are doing
here is adding up all the products $x_{aa'}\,p_{a'b}$ and then subtracting
from that all the products $p_{aa'}\,x_{a'b}$. The first summation
multiplies row $a$ of matrix $X$ with column $b$ of matrix $P$. The
second summation multiplies row $a$ of matrix $P$ with column $b$ of matrix
$X$. The second inner product is then subtracted from the first.
\\\\
Formula (2) does the same thing except that it only uses one summation
and subtracts each element individually, instead of adding up the two
inner products and subtracting them as a whole. But that doesn't matter
since the sum of a difference is the difference of the sums.
\\\\
Formula (3) subtracts the trace of $PX$ from the trace of $XP$. I guess
the easiest way to see this is to recognize
the identity operator: 
\begin{quote}
$
    \sum_{a'}|a'\rangle \langle a'| = \hat{I}
$
\end{quote}
which is embedded in the formula. You can factor out $\hat{I}$ leaving:
\begin{quote}
$
    \sum_{a} 
    \langle a|x p|a\rangle
    - \sum_{a} \langle a|p x|a\rangle
    = \sum_{a} (xp)_{aa} - \sum_{a} (px)_{aa}
    = Tr(XP) - Tr(PX)
$
\end{quote}
(b) Show that the LHS can be made zero if $N$ is finite. I assume that we're
talking about the LHS of formula (3).
You can take the trace of $AB$ (i) by multiplying the rows of
$A$ with the columns of $B$ or (ii) by multiplying the columns of $A$
with the rows of $B$:
\begin{quote}
$
    (i) \sum_{m}\sum_{n} a_{mn}\,b_{nm}\;\;=\;\;
    (ii) \sum_{m}\sum_{n} a_{nm}\,b_{mn}\;\;=\;\;
    (iii) \sum_{m}\sum_{n} b_{mn}\,a_{nm}
$
\end{quote}
But (ii) is
the same as (iii) which is the trace of $BA$. Therefore, in formula (1)
Tr(XP) = Tr(PX) and their difference is zero. This argument uses
discrete summation variables and would not apply - as it stands -
if the variables were continuous. So it does not prove anything, one way
or another, about continuous matrices. {\it I am assuming that the
matrix elements are "c-numbers" and hence commute. The argument I would
make for that is: The trace is basis-independent. Since the matrix is
hermitian it can be diagonalized. In the diagonal basis the entries are
constants. So any operators inside the matrix can ultimately be eliminated
in the trace.
}

\end{document}
