\documentclass{bachw}

% Author and title
\author{Bret Comnes}
\title{PH624 Classical Mechanics Set 2}

\begin{document}
\problem{Goldstein 3.14}

\subproblem{}
\label{sec:circ}
For a circular an parabolic orbits in an attractive $\frac{1}{r}$ potential having the same angular momentum, show that the perihelion distance of the parabola is one-half the radius of the circle.

\solution{}

\subproblem{}
Prove that in the same central force as in part \ref{sec:circ} the speed of a particle at any point in a parabolic orbit is $\sqrt{2}$ times the speed in a circular orbit passing through the same point.
 
\solution{}


\problem{Goldstein 3.19}
A particle moves in a force field described by

\begin{align}
	F(r) &= -\frac{k}{r^2} \exp{\left(-
  \frac{r}{a}
  \right)},
\end{align}

where $k$ and $a$ are positive.

\subproblem{}
Write the equations of motion and reduce them to the equivalent one-dimensional problem.  Use the effective potential to discuss the qualitative nature of the orbits for different values of the energy and the angular momentum.

\solution{}

\subproblem{}
Show that if the orbit is nearly circular, the apsides will davance approximately by $\pi \rho / a$ per revolution, where $\rho$ is the radius of the circular orbit.

\solution{}

\problem{Goldstein 3.23}
Evaluate approximately the ratio of mass of the Sun to that of Earth, using only the lengths of the year and of the lunar month ($27.3$ days), and the mean readii of Earth's orbit ($1.49 \times 10^8$ km) and of the Moon's orbit ($3.8 \times 10^5$ km).

\solution{}

\problem{Goldstein 3.31}
Examine the scattering produced by a repulsive central force $ f = k r^{-3}$.  Show that the differential cross section is given by

\begin{align}
  \sigma(\Theta)d{\Theta} &= 
  \frac
    {k}
    {2E} 
  \frac
    {(1 - x) dx}
    {x^2 (2 - x)^2 \sin{\pi x}},
\end{align}

where $x$ is the ratio of $\Theta / \pi$ and $E$ is the energy.

\solution{}
 
\problem{Goldstein 3.33}
For a particle of mass $m$ is constrained to move under gravity without friction on the inside of a parabolic of revolution whose axis is vertical.  Find the one-dimensional problem equivalent to its motion.  What is the condition on the particle's initial velocity to produce circular motion?  Find the period of small oscillations about this circular motion.

\solution{}

\problem{Goldstein 4.18}

\subproblem{}
Find the vector equation describing the reflection of r in a plane whose unit normal is $\bf{n}$.

\solution{}

\subproblem{}
Show that if $l_i, i = 1,2,3$ are the direction cosines of \bf{n}, then the matrix of the transformations has the elements:

\begin{align}
  A_{i j} &= \delta_{i j} - 2 l_i l_j,
\end{align}

and verify that \bf{A} is an improper orthogonal matrix.

\solution{}

\problem{Goldstein 4.21}

A particle is thrown up vertically with initial speed $v_0$, reaches a maximum height and falls back to ground.  Show that the Coriolis deflection when it again reaches the ground is opposite in direction, and four times greater in magnitude, than the Coriolis deflection when it is dropped at rest from the same maximum height.

\solution{}

\problem{Goldstein 4.24}
A wagon wheel with spokes is mounted on a verticle axis so it is free to rotate in the horizontal plane.  The wheel is rotating with an angular speed of $\omega = 3.0$ radians/s.  A bug crawls out on one of the spokes of the wheel with a velocity of 0.5 cm/s holding on to the spoke with coefficient of friction $\mu = 0.30$.  How far can the bug crawl along the spoke before it starts to slip?


\solution{}

\problem{Goldstein 4.25}
A carousel (counter-clockwise merry-go-round) starts from rest and accelerates at a constant angular acceleration of $0.02$ revolutions/$s^2$.  A girl sitting on a bench on the platform 7.0 m from the center is holding a 3.0 kg ball.  Calculate the magnitude and direction of the force she must exert to hold the ball 6.0 s after the carousel starts to move.  Give the direction with respect to the line from the center of rotation to the girl.

\solution{}


\problem{Goldstein 5.3}
Show directly by vector manipulation that the definition of the moment of inertia as

\begin{align}
  l = m_i(\bf{r}_i \times \bf{n}) \cdot ( \bf{r}_i \times \bf{n})
\end{align}

\solution{}

\problem{Goldstein 5.4}
Derive Euler's equation of motion, Eq.\ (5.39'), from the Lagrange equation of motion, in the form of Eq.\ (1.53), for the generalized coordinate $\psi$.

\solution{}

\problem{Goldstein 5.15}

Find the principal moments of inertia about the center of mass of a flat rigid body the shape of a $45\degree$ right triangle with uniform mass density.  What are the principal axes?

\solution{}

\problem{Goldstein 5.20}
A plane pendulum consists of a uniform rod of length $l$ and negligible thickness with mass $m$, suspended in a vertical plane by one end.  At the other end a uniform disk of the radius $a$ and mass $M$ is attached so it can rotate freely in its own plane, which is the vertical plane.  Set up the equations of motion on the Lagrangian formulation. 

\solution{}

\end{document}
