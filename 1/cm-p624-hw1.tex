\documentclass{jhwhw}
\usepackage{color}
\usepackage{amsmath}
\usepackage{graphicx}
\usepackage{hyperref}
\usepackage{braket}
\usepackage{cancel}
%\everymath{\displaystyle}
%\usepackage{bm}
%\usepackage{setspace}
%\usepackage{verbatim}
%\usepackage[lmargin=2.5cm, rmargin=2.5cm,tmargin=3cm,bmargin=2.5cm]{geometry}
%
% Hyperlink styling
\hypersetup{
colorlinks=true,
linkcolor=blue,
urlcolor=blue
}
%
% This allows you to draw a box around things.
% Wrap whateve you want a box around with:
%\Aboxed{Stuff you want boxed}
\def\@Aboxed#1&#2\ENDDNE{%
  \settowidth\@tempdima{$\displaystyle#1{}$}%
  \addtolength\@tempdima{\fboxsep}%
  \addtolength\@tempdima{\fboxrule}%
  \global\@tempdima=\@tempdima
  \kern\@tempdima
  &
  \kern-\@tempdima
  \fcolorbox{red}{yellow}{$\displaystyle #1#2$}
}
%
% Part of AMS Math
% Lets you create a log-like trace function
\DeclareMathOperator{\Tr}{Tr}
%
% Author and title
\author{Bret Comnes}
\title{PH624 Classical Mechanics Set 1}

\begin{document}
\problem{Goldstein 1.10}
Let $q_1....q_n$, be a set of independent generalized coordinates for a system of $n$ degrees of freedom, with a Lagrangian $L(q,\dot{q},t)$.  Suppose we transform to another set of independent coordinates $s_1....s_n$ by means of transformation equations


(Such a transformation is called a \textit{point transformation}.) Show that if the Lagrangian function is expressed as a function of $s_k$,  $\dot{s}_j$, and $t$ through the equations of transformation, then L satisfies Lagrange's equations with respect to the $s$ coordinates:

In other words, the form of Lagrange's equations is invariant under a point transformation.

\part
Problem part description.

\solution
Solution

\pagebreak[4]

\part
Another part

\solution

\problem{The Second Problem}

\end{document}